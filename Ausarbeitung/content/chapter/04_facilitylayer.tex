\chapter{Facility Layer \label{chap:facilitylayer}}
\section{Komponenten}
\begin{figure}[htbp]
\includegraphics[width=0.99\textwidth]{content/images/04_facilitylayer/facility_layer_model.pdf}
\caption{Die Komponenten der \acl{C2C}}
\label{fig:komponentenfacility}
\end{figure}
Der Facility Layer liegt unterhalb des Applicationlayer und teilt sich in drei Hauptkomponenten auf. Er bietet verschiedene allgemeine Services an die von den Anwendungen des Application Layer verwendet werden können.

\subsection{Application Support}
Der Application Support stellt die Grundfunktionalität für die Anwendungen zur verfügung. Darunter fällt das station lifecycle management, das automatische erkennen von Services sowie das downloaden und initialisieren von neuen Services und noch weitere. Die einzelne Komponenten des Application Support, die auf der \autoref{fig:komponentenfacility} zu sehen sind, werden im folgenden genauer erläutert. 

\subsubsection{Station positioning}
Dieser teil des Application Support verarbeitet die Informationen die aus den verschiedenen Datensammler gewonnen werden wie aus GNSS oder aus Fahrzeugsensoren um die Position des Station zu gewinnen. 
Da diese Angaben sehr genau sein müssen werden die Daten in 3D Position gemessen (latitude, longitude, altitude). Dies können, durch beispielsweise GPS, in Echtzeit zur verfügung stehen. Da es jdeoch Sicherheitsbedingte Anwendungen gibt bei denen bereits über 0.5m Abweichung eine fatale Auswirkung haben kann, werden die Standortinformationen noch einmal über weitere Einheiten verbessert. Dazu gehören z.b. Odometer oder Gyroskop.
Damit diese Support facility ihren Aufgaben Nachkommen kann, ist im \cite{etsi102638} folgende Vorraussetzungen definiert.

\begin{itemize}
\item Benötigt Verbindung zwischen ITS station und einem GNSS system. Bei kurzfristigen Ausfall der Verbindung übernehmen interne mechanische Fahrzeugsensoren wie das Gyroskop oder das Odometer.
\item Benötigt dauerhafte Verbindung zu den Fahrzeugsensoren (direkte Verbindung oder über einen Bus).
\item Benötigt genug Rechenleistung um 3D Positions Angaben sicher zu Verwalten und zu Berechnen.
\end{itemize}

\subsubsection{Mobile station dynamic monitoring}
Das Mobile station dynamic monitoring ist nur für die \acs{VS} verfügbar. Diese Einheit des Facility Layer verwaltet eine große Anzahl an Fahrzeugspezifischen Daten die von den Applicationen und anderen Komponenten des Facility Layers verwendet werden. Sie ist dafür verantwortlich diese Daten immer auf dem neueste  Stand zu halten. 
Diese Einheit benötigt die folgenden Vorraussetzungen.

\begin{itemize}
\item Benötigt zugriff auf Fahrzeug Funktionen wie das Bremssystem, Reifendruck, Stabilitätskontrolle, Geschwindigkeitsregelung etc.
\item Muss verschiedne Parameter filtern können, die für die Applications benötigt werden (z.b Hauptzylinderdruck, Lenkradwinkel, Lenkradwinkeländerungsrate, Gierrate, Fahrzeuggeschwindigkeit oder Beschleunigungssteuerung).
\item Muss über die Möglichkeit besitzen die verschiedene Werte zu aktualisieren, die für die \acl{CAM} benötigt werden.
\end{itemize}

\subsubsection{Station state monitoring \label{facilitylayer_StationStateMonitoring}}
Die Station state monitoring facility überwacht dauerhaft den aktuellen Zustand eines Fahrzeuges. Diese können sich je nach Typ des Fahrzeuges unterscheiden. Bei einem Motorrad würde neben Lichtkontrolle, der Neigungswinkel interessant sein, wobei bei einem Rettungswagen der Status der Sirene mehr Sinn ergibt. Über diese Facility können solche Daten dauerhaft abgerufen werden. 
Dazu benötigt diese die folgenden Punkte.
\begin{itemize}
\item Benötigt Zugriff auf die verschiedene Funktionen die überwacht werden sollen (Motor / Antriebsstrang, Cockpit, Scheibenwischer, Lichtsteuerung, etc.).
\item Muss über die Möglichkeit verfügen diese Funktionen die von den Applications verwendet werden nach aktivität zu filtern.
\item Muss über die Möglichkeit besitzen die verschiedene Werte zu aktualisieren, die für die \acl{CAM} benötigt werden.
\end{itemize}

\subsubsection{Services management \label{facilitylayer_ServicesManagement} }
tr 102 638
\subsubsection{LDM management \label{facilitylayer_LDMManagement}}
tr 102 638
\subsubsection{Messages management \label{facilitylayer_MessagesManagement}}
tr 102 638
\subsubsection{Security access management \label{facilitylayer_AccessManagement}}
tr 102 638
\subsubsection{Time management \label{facilitylayer_TimeManagement}}
tr 102 638
\subsubsection{Time management}
tr 102 638
\subsubsection{Time management}
tr 102 638
\subsection{Information Support}
The information support covers the presentation layer of the OSI reference model and holds the role of
data management. In any ITS system, there will be an abundance of data sources, both mobile and static
ones. These data will mostly be location referenced, time specific and attached with life time value and
with accuracy and reliability parameters. Therefore, fusing data and keeping the information up to date is
one of the challenges of information support. Main entity that supports this is Local Dynamic
Map (LDM) that is able to take data both from different sources and from received ITS messages to build
a data model of the local environment. Furthermore, the information support takes on many functions of
the OSI Presentation Layer. 
\subsubsection{LDM database}
\subsubsection{Data presentation}
\subsubsection{Location referecing}
\subsubsection{Station type/capabilities}

\subsection{Communication Support}
The communication support, which includes the session layer of the OSI Reference model. It will
cooperate with the transport and network layer to achieve the various communication modes required by
the applications. 

\section{CAM\label{sec:cam}}
\ac{ITS} bietet die Möglichkeit, dass \ac{ITS-S} untereinander Statusinformationen austauschen. Für den Austausch von Statusinformationen werden \ac{CAM} versandt. Diese beinhalten Informationen über die Anwesenheit allgemein, über die Position und den grundsätzlichen Status. Gemäß Standard \cite{ts102637-2} soll jede \ac{ITS-S} den Umgang mit diesen Nachrichten beherrschen. Das beinhaltet das Senden, das Empfangen und das Generieren dieser Nachrichten. So kennt jede \ac{ITS-S} ihre aktuellen Nachbarn inklusive ihrer Positionen, Bezwungen und den grundlegenden Sensorinformationen. Die \ac{CAM} werden über den Standard \ac{ITS-G5} übertragen. \ac{CAM} werden mit einer Kadenz zwischen 1 und 10 Hz übertragen, das bedeutet, dass mindestens jede Sekunde eine \ac{CAM} generiert und versendet wird, maximal werden aber 10 \ac{CAM} pro Sekunde generiert und versendet. Der Versand erfolgt über ein Single Hop Broadcast. \ac{CAM} spielen besonders für \ac{IVS} eine Rolle.

Die Verwaltung der \ac{CAM} findet im \ac{CAM} Management statt. Es hat Interfaces zu den anderen Facilities:
\begin{itemize}
	\item Station State Monitoring, \autoref{facilitylayer_StationStateMonitoring} (Bietet den aktuellen Staus der \ac{ITS-S} an)
	\item Mobile Station Dynamic Monitoring (Bietet Echtzeitkinematik der \ac{ITS-S} an)
	\item Time Management, \autoref{facilitylayer_TimeManagement} (Bietet eine global gültige Zeitreferenz für das markieren von Nachrichten an)
	\item Local Dynamic Map Management, \autoref{facilitylayer_LDMManagement}
\end{itemize}

Die anderen Facilities sind in der \autoref{fig:komponentenfacility} eingezeichnet. 

Das Nennen des vollständigen Inhalts einer \ac{CAM} würde den Rahmen dieser Ausarbeitung sprengen. Im Folgenden werden dennoch einige interessante Datenelemente aufgezählt. Sie entstammen dem Standard \cite{ts102637-2}. Dort können auch die Restlichen nachlesen werden.

\begin{itemize}
	\item \textbf{stationCharacteristics: } Dieses Datenelement besteht laut der \ac{ASN.1} Beschreibung aus mindestens drei weiteren Datenelemten:
	\begin{itemize}
		\item mobileItsStation: Kann die \ac{ITS-S} ihre Station ändern?
		\item privateItsStation: Die \ac{ITS-S} wird von keiner Behörde betrieben.
		\item physicalRelevantItsStation: Kann eine andere \ac{ITS-S} diese \ac{ITS-S} rammen?
	\end{itemize}
	\item \textbf{AmbientAirTemperature: } Die Außentemperatur, die die \ac{ITS-S} misst.
	\item \textbf{CrashStatus: } Dieses Datenelement enthält Informationen über einen Unfall, der eine Weiterfahrt verhindert. Beispiele dafür sind das Auslösen der Airbags oder ein Überschlag.
	\item \textbf{CurvatureChange: } Dieses Datenelement beschriebt die Änderung der Kurvenrichtung. Ein positiver Wert bedeutet, dass das \ac{IVS} nach rechts abbiegt.
	\item  \textbf{DangerousGoods: } Dieses Datenelement beschreibt, ob und welches Gefahrgut  das \ac{IVS} transportiert.
	\item \textbf{StationWidth: } Dieses Element beschreibt die zulässige Gesamtbreite der \ac{ITS-S}. Die Breite ist mit einer Feinheit von 0,01m aufgelöst.
\end{itemize}

\subsection{Aufbau der CAM}
Eine \ac{CAM} besteht aus verschiedenen Containern, die wiederum Container enthalten. Ein Container ist...\todo{Aufbau CAM beschreiben}

\section{DEN\label{sec:den}}
\subsection{Beschreibung von DEN \label{facilityLayer_beschreibungDEN}}
\ac{DEN} ist definiert um die \ac{RHW} Anwendungsfälle zu unterstützen.	Die \ac{RHW} Anwendung ist auf \ac{IVS} und \ac{ITS} verteilt. Sie ist eine aktive Straßensicherheit Anwendung. Sie verteilt Informationen über die Straßenverkehrsverhältnisse. Um die \ac{DEN} Informationen zu verteilen gibt es \ac{DENM}. Diese enthalten die relevanten Informationen und informieren die andren \ac{ITS-S}. Sie werden beim Eintreffen eines \ac{RHW} relevanten Ereignisses an alle \ac{ITS-S} verteilt, die durch dieses Ereignis betroffen sind. Während das Ereignis anhält, werden die \ac{DENM} verteilt. Das Verteilen endet nach einer festgelegten Zeit oder durch eine weitere \ac{DENM}, die eine andere beendet. In einer \ac{ITS-S}, die eine \ac{DENM} empfängt, wird geprüft, ob die Information für den Benutzer relevant ist. Abhängig der Relevanz wird sie angezeigt oder nicht. Ein Beispiel für eine \ac{DENM}, die nach einer festen Zeit ungültig wird, ist eine Notbremsung eines \ac{IVS}. Durch die Bremsung entsteht eine temporäre Gefahr. 


\subsection{DEN Management}
Auf der \autoref{fig:darstellungDenServices} sind die Komponenten von \ac{DEN} dargestellt. Darauf zu erkennen ist, dass der \ac{DEN} Basic Service aus das \ac{DEN} Management und die \ac{LDM} besteht. Die restlichen Komponenten unterstützen den Service. 

\begin{figure}[htbp]
	\includegraphics[width=0.75\textwidth]{content/images/04_facilitylayer/denServices.pdf}
	\caption{Service Komponenten von DEN \cite{ts102637-3}}
	\label{fig:darstellungDenServices}
\end{figure}


Das \ac{DEN} Management  ist für das Steuern der \ac{DENM} Übertragung verantwortlich. Dazu gehören das Gerieren, das Aktualisieren und das Beenden der Übertragung. Es regelt das \ac{DENM} Protokoll. Das schließt die Einhaltung des korrekten Nachrichtenformats und der korrekten Semantik ein. Zum erstellen von \ac{DENM} hat das \ac{DEN} Management Interfaces zu den \ac{RHW} Anwendungen und den anderen Facilities. Die \ac{DENM} werden in diesem Services auch aktuell gehalten. Veraltete \ac{DENM} werden gelöscht, bzw. für ungültig erklärt. Eine \ac{DENM}, die im \ac{DEN} Management erzeugt wird, muss mindestens den Typ des entdeckten Events und dessen Position enthalten. Zusätzlich muss sie die Eintrittszeit des Events und die voraussichtliche Dauer des Events enthalten. Kann keine voraussichtliche Dauer genannt werden, muss sie geschätzt werden. Das betroffene Gebiet muss auch genannt werden. Bei dem Gebiet, bzw. relevante area, handelt es sich um einen oder mehrere Straßenabschnitte, in denen sich die \ac{ITS-S} befinden die von dem Event betroffen werden. Ein Gebiet kann aber auch eine Richtung sein.

\subsection{DEN Basic Service}
Der \ac{DEN} Basic Service ist für die Verarbeitung von \ac{DENM} verantwortlich. Die Verarbeitung beinhaltet auf der sendenden Seite das Auslösen von \ac{DENM} und auf die der empfangenden Seite das Empfangen der \ac{DENM}, sowie deren Bereitstellen für die Benutzung in den \ac{ITS-S} Anwendungen. Das Auslösen kann in mehrere Tätigkeiten unterteilt werden und wird im Folgenden genauer erklärt. Der \ac{DENM} Basic Service kann auch die Funktionalität des Weiterleitens von \ac{DENM} anbieten. Der \ac{DEN} Basic Service besteht wiederum aus fünf weiteren Komponenten. Er hat fünf Interfaces. Die Komponenten und Interfaces sind in der \autoref{fig:darstellungDenBasicServiceKomponenten} abgebildet.

\subsubsection{Interfaces}
\todo{Interfaces vom DENM Basic Service beschreiben}

\subsubsection{Komponenten}
Die Komponenten des \ac{DENM} Basic Service erfüllen die Aufgaben des Basic Service.

\paragraph{Endode DENM \label{facilitylayer_EncodeDENM}}
Diese Komponente konstruiert eine \ac{DENM}. Dazu werden die zu übertragenden Daten kodiert. In dieser Komponente werden lediglich die vorhandenen Daten zu einer \ac{DENM} zusammengefasst. Die Kodierung erfolgt mithilfe von \ac{ASN.1} und ist im Standard \cite{en302637-3} beschrieben. 

\paragraph{Decode DENM}
Diese Komponente dekodiert die empfangenen \ac{DENM} und stellt somit die enthaltenen Daten bereit.

\paragraph{DENM transmission management}
Das Transfusion Management erfüllt die Aufgaben, die für das Versenden von \ac{DENM} nötig sind. Das Versenden beginnt bei der Erstellung einer \ac{DENM}. Die Erstellung wird von einer Application angefordert. Nachdem die \ac{DENM} erstellt wurde, wird sie von der Komponente \ref{facilitylayer_EncodeDENM} kodiert. Neben der Erstellung einer neuen \ac{DENM} können auch vorhandene \ac{DENM} aktualisiert werden. Dies geschieht auch in diesem Layer und läuft analog zu dem Erstellen neuer \ac{DENM}. Das Beenden von \ac{DENM} Übertragungen und die wiederholte Übertragung wird auch in dieser Komponente gemacht.

\paragraph{DENM reception management}
Das Empfangsmanagement stellt die Funktionalitäten des \ac{DENM} Empfangs bereit. Beim Empfangen werden die \ac{DENM} überprüft und verworfen, wenn sie ungültig sind. Wenn sie gültig sind werden sie in die Message Table eingetragen. Die Message Table speichert die empfangenen \ac{DENM}. Dabei werden \ac{DENM} mit der gleichen actionId nur einmal gespeichert. Es wird immer nur die aktuelle \ac{DENM} mit einer actionId gespeichert. Die message table muss mindestens aus den Folgenden Spalten bestehen:

\todo{Prüfen, ob die genannten Werte noch erklärt werden. Wenn nicht $\Rightarrow$ hier erklären}
\begin{itemize}
	\item actionID: Die actionID der \ac{DENM}, auf die sich die Zeile bezieht.
	\item referenceTime: Diese Spalte erhält die referenceTime der \ac{DENM}, die als letztes empfangen wurde. 
	\item termination: Diese Spalte enthält die termination Werte der gespeicherten \ac{DENM}
	\item detectionTime: Diese Spalte enthält die detectionTime Werte der gespeicherten \ac{DENM}
\end{itemize}
 
Das \ac{DENM} reception management stellt stellt die die Daten der empfangen \ac{DENM} den anderen Komponenten des Facility Layers und den Anwendungen zur Verfügung.

\paragraph{DENM Keep Alive Forwarding}
Das Keep Alive Forwaring implementiert die das \ac{DENM} Forwarding Protokoll zu


\begin{figure}[htbp]
	\includegraphics[width=0.75\textwidth]{content/images/04_facilitylayer/denBasicServiceKomponenten.pdf}
	\caption{Komponentendiagramm mit den Komponenten des DEN Basic Service \cite{en302637-3}}
	\label{fig:darstellungDenBasicServiceKomponenten}
\end{figure}

\subsection{Verbreitung der DENM}
Eine \ac{DENM} soll an so viele \ac{ITS-S} des gleichen Gebiets wie möglich verteilt werden. Das schließt auch \ac{ITS-S} ein, die während der Eventdauer das Gebiet erreichen. Auch wenn eine \ac{ITS-S} keine direkte Verbindung zu der auslösenden \ac{ITS-S} hat, soll sie die \ac{DENM} empfangen. Deswegen werden einige Parameter von der \ac{RHW} Applikation selber definiert. Die Liste dieser Parameter beginnt mit dem Zeitintervall zwischen den \ac{DENM}, die von der gleichen \ac{ITS-S} ausgesendet werden. Neben der Frequenz wird auch die maximale Latenz definiert, die für das Senden der \ac{DENM} benötigt wird und die Priorität der \ac{DENM}. Die Latenz beschreibt die Zeit, die \ac{DENM} vom Facility Layer der aussendenden \ac{ITS-S} zum Networking and Transport Layer der empfangenden \ac{ITS-S} zu senden. Die Priorität wird nicht in der \ac{DENM} selber festgelegt, da sie von unteren Layern übertragen wird. Das Zielgebiet für die \ac{DENM} wird auch von der \ac{RHW} Application festgelegt.


\begin{figure}[htbp]
	\includegraphics[width=0.99\textwidth]{content/images/04_facilitylayer/denVersendenLayerUeberblick.pdf}
	\caption{Datenfluss einer DENM über die Layer \cite{en302637-3}}
	\label{fig:darstellungDenVerteilenLayer}
\end{figure}

Die \autoref{fig:darstellungDenVerteilenLayer} zeigt, wie sich eine \ac{DENM} über die verschiednen Layer ausbreitet. In diesem Prozess sind drei \ac{ITS-S} beteiligt. Es gibt eine sendende \ac{ITS-S}, eine Empfangende und eine, die nur weiterleitet. Zu sehen ist, dass die Verarbeitung der \ac{DENM} im Application Layer beginnt und endet. Am Weiterleiten sind nur die unteren Layer, bis einschließlich der Facilities Layer beteiligt. Die Entscheidung, ob und wie weitergeleitet wird wird im \ac{DEN} Basic Service getroffen. 

\subsection{Die Aktualisierung von DENM}
\todo{DENM Update schreiben} 
 
\subsection{Die Beendung des Events}
Bereits im \autoref{facilityLayer_beschreibungDEN} wurde beschrieben, dass ein Event entweder durch den Ablauf einer Zeit oder durch eine \ac{DENM}, die das Event beendet, beendet werden. Eine beendende \ac{DENM} kann entweder eine Annullierende oder eine Negierende sein. 

Die annullierende \ac{DENM} wird von der \ac{ITS-S} versendet, die das Ereignis ursprünglich bekannt gemacht hat. Sie wird versendet, wenn die \ac{ITS-S} feststellt, dass das Event beendet wurde.  Die annullierende \ac{DENM} ist eine \ac{DENM} mit einer speziellen Datenversion.

Die Negierende \ac{DENM} wird von \ac{ITS-S} gesendet, die das Event ursprünglich nicht angekündigt haben. Sie haben bereits \ac{DENM} zu dem Event empfangen und festgestellt, dass das Event nicht weiter existiert. Eine negierende \ac{DENM} enthält ein Negationsflag. 

Wenn eine beendende \ac{DENM} aus einer vertrauenswürdigen Quelle empfangen wurde, löscht das \ac{DEN} Management der empfangenden \ac{ITS-S} alle anderen \ac{DENM} zu dem gleichen Event. Die beendende \ac{DENM} muss, genau wie die Ankündigenden, eine Zeit lang weiter verteilt werden. Diese Zeit wird von der \ac{RHW} Application bestimmt.

\subsection{Aufbau der DEMN}
Eine \ac{DENM} besteht aus mehren Containern. Die Container sind in einer festen Reihenfolge angeordnet.   

\begin{figure}[htbp]
	\includegraphics[width=0.95\textwidth]{content/images/04_facilitylayer/denmAufbau.pdf}
	\caption{Aufbau einer DENM \cite{ts102637-3}}
	\label{fig:darstellungdenmAufbau}
\end{figure}

\subsubsection{Management Container}
Der Management Container enthält Management Daten des entdeckten Events.

\subsubsection{Situation Container}
Der Situation Container enthält Daten zu dem entdeckten Event.

\subsubsection{Location Container}
Der Location Container enthält die Informationen zur Position des entdeckten Events.



Neben der Nutzung von \ac{DENM} für Anwendungen der Verkersicherheit können sie auch für Verkehreffizienzanwendungen genutzt werden. In diesem Fall müssen die \ac{DENM} von den \ac{IVS} auch über lange Distanzen zu den \ac{ICS} geroutet und transportiert werden. 



\section{SPaT\label{sec:spat}}
\ac{SPaT} ist eine Nachricht, die den Zustand einer Straßenkreuzung beschreibt. Die Beschreibung enthält Staus Informationen der \ac{VBA} und Informationen über die Kreuzung selber. \ac{SPaT} ist noch nicht von der \ac{ETSI}, bzw. \ac{ISO} standardisiert. Aus diesem Grund stammen die folgenden Informationen hauptsächlich aus den U.S. amerikanischen Quellen \cite{usSpat} und \cite{usCaliforniaSpat}. 

Für \ac{SPaT} gibt es zwei grundsätzliche Betriebsarten. Die erste Betriebsart ist eine feste \ac{VBA} Schaltung. Diese wird den \ac{IVS} über \ac{SPaT} Nachrichten mitgeteilt. Anhand dieser Informationen kann das \ac{IVS} selber die Ampelzeiten berechnen und benötigt keine  weiteren \ac{SPaT} Nachrichten. Die zweite Betriebsart ist, dass die Ampelzeiten aufgrund externer Ereignisse beeinflusst werden können. Externe Ereignisse sind beispielsweise Fußgänger, verkehrsabhängige Schaltungen oder periodisierte \ac{IVS}, wie z.B. Rettungswagen. Bei der zweiten Betriebsart muss beachtet werden, dass zu schnelle Änderungen vermieden werden müssen. Die \ac{IVS} benötigen eine bestimmte Zeit zur Reaktion, so benötigt beispielsweise ein Auto die Reaktionszeit des Fahrers und einen Bremsweg. Wird die Fahrspur dieses Autos gesperrt und andere Fahrspuren dafür freigegeben, wenn das Auto sich schon fast vor der Haltelinie befindet, so ist eine Bremsung vor der Kreuzung nicht mehr möglich. Daraus resultiert ein Sicherheitsrisiko, da die andere Fahrspur nicht für die anderen Fahrzeuge garantiert frei ist. \todo{Rausfinden, ob die \ac{IRS} die Position des \ac{IVS} verwertet und daraus die Zeiten generiert. S.29 US Department.. -> Ich glaube die verwendet die Daten}

Auf dieser Grundlage können weitere Dienste angeboten werden. So können Abbiegeassistent Systeme oder Vorzugsschaltungen realisiert werden. 

 Abbiegeassistent Systeme unterstützen \ac{IVS} beim Abbiegen in Kreuzungen. Anhand der Positions- und Geschwindigkeitsdaten der \ac{IVS} kann die \ac{IRS} berechnen, ob ein Abbiegen nach rechts oder nach links sicher ist. Neben der Positionsdaten können auch Erfassungssysteme genutzt werden, die in die Infrastruktur integriert sind. Beispiele hierfür können \ac{Radar} oder \ac{Lidar} Systeme genutzt werden. Der Nachteil von rein \ac{ITS} basierten Systemen ist, dass nicht \ac{ITS} kompatible Fahrzeuge nicht erkannt werden. Das bedeutet, dass die \ac{IRS} eine freie Kreuzung erkennt und meldet, obwohl nicht \ac{ITS} basierende Fahrzeuge über die Kreuzung fahren. 
 
Die Bevorzugung von Fahrzeugen bietet sich für die Empfehlung \cite{usSpat} für Notfallfahrzeuge, für Fahrzeuge des Nahverkehrs und für den Frachtverkehr an. Durch die Bevorzugung können diese \ac{IVS} eine grüne Welle erreichen. Das bedeutet, dass entweder Grünphasen verlängert werden oder Rotphasen unterbrochen werden. Sofern nicht bereits durch andere Nachrichten, wie \ac{CAM} geschehen, fordert das \ac{IVS} bei der \ac{IRS} eine Bevorrangung an, wenn anhand der \ac{SPaT} erkennt, dass die Kreuzung nicht frei ist. Die \ac{IVS} beantwortet diese Anfrage im positiven Fall mit einer weiteren \ac{SPaT} Nachricht, die eine freie Kreuzung signalisiert.   

Inhalt einer \ac{SPaT}:
\begin{longtable}{|c|c|}
 \hline
\textbf{Datenelemente} & \textbf{Typ/Größe}\\
 \hline
 \hline
Object ID & Unsigned 8 Bit Integer\\
 \hline
 Object Size& Unsigned 8 Bit Integer\\
 \hline
 Apptoach ID & Unsigned 8 Bit Integer\\
 \hline
 Signal Phase Indication &  32 Bit Bitmask \\
 \hline
 Countdown Timer Confidence & 32 Bit Bitmask \\
 \hline
 Time to Signal Phase Change (Countdown Time) & Unsigned 16 Bit Integer \\
 \hline
 Yellow Duration & Unsigned 8 Bit Integer\\
 \hline
 \caption{Inhalt einer SPaT Nachricht \cite{usSpat}}
 \label{tab:facilitylayer_inhaltSpatNachricht}
 \end{longtable}
 
\todo{Werte der Tabelle beschreiben oder den Mist löschen}
\begin{figure}[htbp]
	\includegraphics[width=0.60\textwidth]{content/images/04_facilitylayer/spatKreuzung-Anfahrtsstreifen.pdf}
	\caption{Kreuzung mit Darstellung der Approach ID \cite{usSpat}}
	\label{fig:darstellungKreuzung}
\end{figure}




\section{TOPO\label{sec:topo}}
