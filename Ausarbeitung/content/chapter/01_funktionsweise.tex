\chapter{Funktionsweise \label{chap_funktionsweise}}
\ac{ITS} besteht aus verschiedenen Komponenten. Diese Komponenten können mobil oder stationär sein. Die Grafik \ref{fig:funktionsweise_komponentenueberblick} gibt einen Überblick über die in \ac{ITS} verwendeten Komponenten. Jede dieser Komponenten enthält eine \ac{ITS} Station. In diesem Abschnitt mischen sich Beschreibungen aus bereits realisierten Projekten und dem Standard \cite{etsi2010302}. Der Standard definiert lediglich Mindestfunktionalitäten, in den Projekten sind diese implementiert, dort können aber Funktionalitäten implementiert sein, die über den Standard hinaus gehen.

\begin{figure}
\includegraphics[width=0.99\textwidth]{content/images/01_funktionsweise/ueberblick-ITS-subsystems.pdf}
\caption{Überblick über die Komponenten \cite{etsi2010302}}
\label{fig:funktionsweise_komponentenueberblick}
\end{figure}

\section{ITS Station Reference Architecture}
\todo{Evtl. kommt das in das Chapter Architektur}
Um das folgende Kapitel zu verstehen muss die \ac{ITS} Station Reference Architecture betrachtet werden. Eine Referenzarchitektur beschreibt ein allgemeines Modell einer Architektur. Das bedeutet, dass basierend auf dieser Architektur verschiedene Implementierungen existieren können. 

Die \ac{ITS} Station Reference Architecture unterscheidet sich grundlegend von bekannten Architekturen. Verglichen mit dem \ac{OSI} Modell kann man Gemeinsamkeiten erkennen:
\begin{itemize}
	\item Trennung der einzelnen Layer
	\item Definition von Service Primitiven zwischen den Layern
\end{itemize}

Bei der genaueren Beschreibung der Layer fällt auch auf, dass die Beschreibungen sich stellenweise auf das \ac{OSI} Modell beziehen. Der Hintergrund ist, dass die \ac{ITS} Station Reference Architecture an das \ac{OSI} Modell angelehnt ist. Es wurde aber um die Besonderheiten von \ac{ITS} erweitert.

Es gibt jedoch einen gravierenden Unterschied: In der \ac{ITS} Station Reference Architecture sind Cross Layer vorgesehen. Das \ac{OSI} Referenzmodell ist wasserfallartig aufgebaut. Das bedeutet, dass die einzelnen Layer übereinander angeordnet sind. Jeder Layer hat jeweils nur zu dem direkt über- und unterliegenden Layer eine Schnittstelle. Cross Layer sind Layer, die in mehrere dieser Schichten Schnittstellen haben. Im Fall der \ac{ITS} Station Reference Architecture sind das die Layer \glqq Management\grqq~ und \glqq Security\grqq. Sie haben Schnittstellen, bzw. Primitiven in alle anderen Layer. 

Möglichkeit Aufgaben dieser Layer sind:
\subsection{Management Layer}
Beschrieben in \cite{etsi1027232}
\todo{Management Layer genauer beschreiben}

\subsection{Security Layer}

\subsection{Access}
Der Access Layer entspricht den \ac{OSI} Layern 1 und 2. 

\subsection{Networking \& Transporting}
Der Networking \& Transporting Layer entspricht den \ac{OSI} Layern 3 und 4.

\subsection{Facilities}
Der Facilities Layer entspricht den \ac{OSI} Layern 5, 6 und 7

\subsection{Applications}


\begin{figure}
\includegraphics[width=0.75\textwidth]{content/images/01_funktionsweise/stationReferenceArchitecture.pdf}
\caption{Darstellung der ITS Station Reference Architecture \cite{etsi2010302}}
\label{fig:funktionsweise_referenceArchitecture}
\end{figure}

\section{Untereinheiten}
Die Untereinheiten sind sind in sich geschlossene Einheiten, die in den \ac{ITS} Komponenten vorhanden sind. Sie werden in diesem Abschnitt beschrieben. Zusätzlich werden die mindestens definierten Funktionalitäten genannt.

  
\subsection{Roadside ITS-S Gateway}
Die Funktionsweise des Roadside ITS-S Gateway ergibt sich aus der Grafik \ref{fig:funktionsweise_itsGateway}. Die Aufgabe ist die gleiche, wie bei den meisten Gateways. Es verbindet unterschiedliche Protokollstocks miteinander. In diesem Fall werden das \ac{ITS} interne Netzwerk und ein proprietäres Netzwerk miteinander verbunden. Das proprietäre Netzwerk kann beispielsweise ein IP basierendes Netzwerk sein.

\begin{figure}[h]
\includegraphics[width=0.75\textwidth]{content/images/01_funktionsweise/layer_gateway.pdf}
\caption{Überblick über die Layer eines ITS Gateways \cite{etsi2010302}}
\label{fig:funktionsweise_itsGateway}
\end{figure}

\subsection{ITS-S Host}
Der ITS-S Host beinhaltet mindestens die ITS-S Anwendungen und die Funktionalität der ITS  Station Reference Architektur, die für die  ITS-S Anwendungen gebraucht wird. Konkret sind das der \ac{ITS} Network Layer und die Anwendungsebene. 


\subsection{ITS-S Router}


\begin{figure}
\includegraphics[width=0.99\textwidth]{content/images/01_funktionsweise/layer_router.pdf}
\caption{Überblick über die Layer eines ITS Hosts \cite{etsi2010302}}
\label{fig:funktionsweise_layerHost}
\end{figure}

\subsection{ITS-S Border Router}
\begin{figure}
\includegraphics[width=0.99\textwidth]{content/images/01_funktionsweise/layer_borderRouter.pdf}
\caption{Überblick über die Layer eines ITS Border Routers \cite{etsi2010302}}
\label{fig:funktionsweise_borderRouter}
\end{figure}

\section{Personal subsystem and station}
\ac{PSS} stellen die Funktionalitäten von \ac{ITS} in Geräten zur Verfügung, die in der Hand gehalten werden können. Der Standard nennt hierzu \ac{PDA} oder Mobiltelefone als Beispiel. Sie können als eigenständige Komponente dienen, oder als Teil einer anderen Komponente arbeiten.

\section{ITS Central Station}
Die \ac{ICS}, oder Central \ac{ITS} subsystem and station ist eine zentrale Komponente im \ac{ITS} System. Sie bietet die Funktionalität an, um die Komponenten des zentralen Systems an das \ac{ITS} 

\section{ITS Roadside Station}
Die Kommunikation ist nicht auf die Kommunikation von Fahrzeugen untereinander beschränkt. Eine Kommunikation zwischen Fahrzeugen und Verkehrsinfrastruktur ist ebenfalls möglich. Diese Kommunikation wird über \ac{IRS} oder \ac{RSU} abgewickelt. Da sie den Informationsfluss zwischen \ac{IVS} und \ac{ICS} ermöglicht, hat sie einen hohen Stellenwert im System. \ac{IRS} werden im Normalfall in bereits vorhandene Infrastruktur integriert. Hierfür bieten sich beispielsweise Ampeln oder sonstige Verkehrsleitsysteme an. 

Die \ac{IRS} beherrscht zwei grundlegend unterschiedliche Verbindungsprotokolle. Über das verbindungslose ITS-G5 kann die \ac{IRS} Verbindungen zu den \ac{IVS} aufbauen. Die Verbindung zu den \ac{ICS} erfolgt über TCP/IP. 

Neben der Funktion als reine Schnittstelle zwischen \ac{IRS} und \ac{IVS} kann die \ac{IRS} die empfangenen Daten aufbereiten, bzw. ein FunctionFramework zur Verfügung stellen, auf dem Applikationen ausgeführt werden können.


Beispiele für Applikationen der \ac{IRS} sind:
\begin{itemize}
	\item Store and Forward von Ereignisinformationen \ac{DENM}.
	\item Weiterleitung von Ereignisinformationen an Versuchszentrale (Testzentrale)
	\item Aggregation von empfangenen Fahrzeugdaten zur Verbesserung der Wetter- und Verkehrslageerfassung. 
	\item Neue Anwendungen bzgl. der Interaktion zwischen Fahrzeug und LSA.
	\item  Kreuzungsassistenz sowie Assistenz im Baustellenbereich.
	\item Verteilung von Daten der ergänzenden Dienste aus der Versuchszentrale an die Fahrzeuge.
	\item Versendung von Daten zur Kreuzungstopologie.
\end{itemize}

Diese Beispiele sind aus einem Projektergebnis von \ac{SIMTD} entnommen (\cite{simtd-D12.1}). Im Standard sind die Funktionalitäten einer \ac{IRS} nicht genauer beschrieben. 




\section{IVS}


