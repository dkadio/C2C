\chapter{Architektur  \label{chap_archtitektur}}
Die Netzwerkarchitektur von \ac{ITS} Stations umfasst sowohl interne als auch externe Netzwerke. Laut Standard \cite{etsi302636-3}, bzw. Standard \cite{etsi102636-3} sind dabei folgende externe Netzwerke erfasst:

\begin{itemize}
 	\item ITS ad hoc network.
	\item Access network (ITS access network, public access network, private access network).
	\item Core network (e.g. the Internet).
\end{itemize}

Auf der Grafik \ref{fig:architektur_ueberblickNetzwerke} sind die verschiedenen Netzwerke visualisiert. Diese Art der Darstellung entspricht der höchsten Abstraktionsebene. Die verschiedenen Netzwerke sind in der Grafik als Wolken dargestellt. Neben den Netzwerken sind auch die Verbindungen visualisiert.


Zusätzlich zu den hier beschriebenen Netzwerken kann eine \ac{ITS} Station ein eigenes Netzwerk, das die Teilkomponenten der \ac{ITS} Station verbindet, betreiben. Die verschiedenen Netzwerke werden benötigt, damit alle Dienste mit ihren verschiedenen Anforderungen bedient werden können.

\begin{figure}[h]
	\includegraphics[width=0.99\textwidth]{content/images/02_architektur/uebersichtExterneNetzwerke.pdf}
	\caption{Überblick über die externen Netzwerke \cite{etsi302636-3}}
	\label{fig:architektur_ueberblickNetzwerke}
\end{figure}

\section{Übersicht über die verschiedenen Netzwerke}
Dieser Abschnitt soll lediglich eine Übersicht über die verwendeten Netzwerke geben. Eine weitere Erklärung der Netzwerke findet an dieser Stelle nicht statt und ist nicht Gegenstand dieser Ausarbeitung. Die Netzwerke dürfen auch nicht isoliert betrachtet werden. Im Abschnitt \ref{funktionsweise_funktionaleKomponenten} wurden funktionale Komponenten mit Routingfunktionalitäten vorgestellt. Diese können die Netze und somit ihre Vorteile, bzw. ihre Dienste, miteinander verbinden.

Selbstverständlich benötigen die \ac{ITS} Stations Zugang zu einem der im Folgenden aufgeführten Netze. Der Zugang zum Core Network \ref{achitektur_coreNetwork} erfolgt über eins der anderen Netze. 

\subsection{ITS Ad Hoc Network\label{achitektur_adHocNetwork}}
Das \ac{ITS} Ad Hoc Netzwerk ist das Netzwerk für die Kommunikation zwischen \ac{IRS}, \ac{IVS} und \ac{PSS}. Die Kommunikation findet über die Luftschnittstelle statt. Sie ist in ihrer Reichweite begrenzt, dafür ist sie mobil einsetzbar. Die Drahtlose Kommunikation wird im Normalfall über den Standard ITS-G5 ermöglicht.


\subsection{ITS Access Network \label{architektur_itsAccessNetwork}}
\tocheck{Nicht die blasseste Ahnung ob das mit den Access Networks stimmt}
ITS Access Network werden zur Vernetzung von \ac{ITS} Komponenten verwendet. Diese Netzwerke bieten den Zugang für die entsprechenden \ac{ITS} Services. Sie werden als eigene Netzwerke realisiert. \ac{ITS} Stations werden durch Access Networks verbunden. Das bedeutet, dass \ac{IRS} untereinander über Access Networks verbunden sein können, es können aber auch Stations, die normalerweise Ad Hoc miteinander kommunizieren dieses Netz nutzen. 

\subsection{Public Access Network}
Ein Public Access Network ermöglicht den Zugang in öffentlich zugängliche Mehrzwecknetzwerke. Dieses Netzwerk kann beispielsweise dazu genutzt werden, um \ac{ITS} Stations mit dem Core Netzwerk zu verbinden. 

\subsection{Private Access Network}
Ein Private Access Network reguliert den Zugang durch die Teilnehmer. Die angebotenen Datendienste stehen nur einer bestimmten Gruppe von Nutzern zur Verfügung. Mit Private Access Networks besteht die Möglichkeit, eine gesicherte Verbindung in ein anderes Netzwerk aufzubauen. So kann beispielsweise ein \ac{IVS} auf das Intranet einer Firma zugreifen. 

\subsection{Core Network \label{achitektur_coreNetwork}}
Das Core Network ist ein Verbindungsnetz. Es hat keine \ac{ITS} Funktionalitäten und wird im Standard auch nicht weiter spezifiziert. Es wird in Verbindung mit den Public Access Network dazu genutzt, traditionelle Dienste, wie Internet oder Email, anzubieten.
 
\begin{figure}
	\includegraphics[width=0.75\textwidth]{content/images/02_architektur/netzwerkSzenario.pdf}
	%\vspace{2cm}
	\includegraphics[width=0.75\textwidth]{content/images/02_architektur/verbindungenNetzwerkSzenario.pdf}
	\caption{Netzwerkszenario mit dazugehöriger Implementierung \cite{etsi302636-3}}
	\label{fig:architektur_netzwerkSzenario}
\end{figure}

Die Grafik \ref{fig:architektur_netzwerkSzenario} stammt aus dem Standard \cite{etsi302636-3}. Dort ist im oberen Teil der Grafik ein Szenario beschrieben, welche Netzwerke miteinander verbunden sein können.  Zu erkennen ist, dass die Netzwerke ITS Ad Hoc Netzwerk \ref{achitektur_adHocNetwork}, ITS Access Network \ref{architektur_itsAccessNetwork} und das Core Network  \ref{achitektur_coreNetwork} miteinander verbunden sein sollen.

Der untere Teil der Grafik zeigt eine Implementierungsmöglichkeit dieses Szenarios. Die hellen Rechtecke beschreiben die Komponenten, die in dieser Implementierung im System integriert sind, das dunkle Rechteck beschreibt die funktionale Komponente, die in diesem System beteiligt ist. Die Linien sind mit dem Typ des Netzwerks, welches sie repräsentieren beschriftet und zusätzlich mit dem Network Reference Point, den sie benutzen, beschriftet.
\todo{Sollen wir hier noch was zum Thema Network Reference Point schreiben? Ich glaube aber, dass die noch in den Layern kommen}


Auflistung und kurze Beschreibung der genutzten Network Reference Points:
\begin{itemize}
	\item \textbf{RA: } Reference Point zwischen \ac{IVS} über das ITS Ad Hoc Network
	\item \textbf{RB: } Reference Point zwischen \ac{IVS} und \ac{IRS} über das ITS Ad Hoc Network
	\item \textbf{RC: } Reference Point zwischen \ac{PSS} über das ITS Ad Hoc Network
	\item \textbf{RD: } Reference Point zwischen \ac{PSS} und \ac{IRS} über das ITS Ad Hoc Network
	\item \textbf{RE: } Reference Point zwischen \ac{IVS} und \ac{PSS} über das ITS Ad Hoc Network
	\item \textbf{RF: } Reference Point zwischen \ac{IRS} über das ITS Access Network
	\item \textbf{RG: } Reference Point zwischen \ac{IRS} und einem ITS-S Border Router \footnote{Der Border Router muss nicht explizit aufgeführt werden, da er als funktionale Komponente Teil einer Komponente ist. \label{ftn:borderRouter}} über das ITS Access Network 
	\item \textbf{RH: } Reference Point zwischen \ac{ICS} und ITS-S Border Router \footref{ftn:borderRouter} über das Core Network		
\end{itemize}

Erkennbar ist in dieser Implementierung, dass sich für mobile Stations Ad Hoc Netzwerke verwendet wurden. Diese haben den Vorteil, dass sie bereits in der Spezifikation mit der Luftschnittstelle ITS-G5 ausgestattet sind, was eine Mobilität erst ermöglicht. Was auch erkennbar ist, ist, dass die reinen ITS Netzwerke durch einen Border Router vom Core Network getrennt sind. Auch wenn hier nicht explizit aufgeführt, die \ac{ICS} benötigt in diesem Fall auch einen Border Router.

 
\section{ITS Station Reference Architecture}
\todo{komplette Section ITS Station Reference Architecture überarbeiten}
Eine Referenzarchitektur beschreibt ein allgemeines Modell einer Architektur. Das bedeutet, dass basierend auf dieser Architektur verschiedene Implementierungen existieren können. 

Die \ac{ITS} Station Reference Architecture unterscheidet sich grundlegend von bekannten Architekturen. Da sie während der Entwicklung an das \ac{OSI} Modell angelehnt war, ergeben sich einige Parallelen:
\begin{itemize}
	\item Trennung der einzelnen Layer
	\item Definition von Service Primitiven zwischen den Layern
	\item Die Standards beziehen die Layer auf die \ac{OSI} Layer. 
\end{itemize}


Obwohl das \ac{ITS} Station Reference Protocol bei der Entwicklung an das \ac{OSI} Modell angelehnt wurde gibt es jedoch einen gravierenden Unterschied: In der \ac{ITS} Station Reference Architecture sind Cross Layer vorgesehen. Das \ac{OSI} Referenzmodell ist wasserfallartig aufgebaut. Das bedeutet, dass die einzelnen Layer übereinander angeordnet sind. Jeder Layer hat jeweils nur zu dem direkt über- und unterliegenden Layer eine Schnittstelle. Cross Layer sind Layer, die in mehrere dieser Schichten Schnittstellen haben. Sie erweitern die vorhanden Layer in horizontaler Richtung. Im Fall der \ac{ITS} Station Reference Architecture sind das die Layer \glqq Management\grqq~ und \glqq Security\grqq. Sie haben Schnittstellen, bzw. Primitiven in alle anderen Layer. 

\begin{figure}
	\includegraphics[width=0.75\textwidth]{content/images/02_architektur/stationReferenceArchitecture.pdf}
	\caption{Darstellung der ITS Station Reference Architecture \cite{etsi2010302}}
	\label{fig:funktionsweise_referenceArchitecture}
\end{figure}

\section{Horizontal Layer}
Dieser Abschnitt beschreibt die Layer, die klassisch übereinander angeordnet sind. Die Layer und ihre Funktionen entsprechen den Layern des \ac{OSI} Modells. Sie sind aber anders aufgeteilt.

\subsection{Access}
\todo{Noch etwas über die Channel herausfinden und schreiben}
Der Access Layer von ITS entspricht den \ac{OSI} Layern 1 und 2. Er besteht aus zwei Subaltern und hat drei Interfaces, bzw. \ac{SAP}. Die Sublayer sind der \glqq Data Link Layer (DLL)\grqq~und der \glqq Physical Layer (PHY)\grqq. Der DLL kann weiter in den \glqq Medium Access Control (MAC)\grqq~und den \glqq Logical Link Control (LLC)\grqq~Layer unterteilt werden. Zusätzlich zu den Subalayern hat der Access Layer ein Layer Management. Dieses verwaltet die Sublayer. Es arbeitet nur nur im Access Layer und darf nicht mit dem  Management Layer \ref{architektur_managementLayer} verwechselt werden.

Die \ac{SAP} sind:
\begin{itemize}
	\item \textbf{SAP-IN: } Als \ac{SAP} zu dem nächst höheren Layer  Networking \& Transporting \ref{architektur_networkingTransporting}
	\item \textbf{SAP-SI: } Als \ac{SAP} zu dem Cross Layer Security Layer \ref{architektur_securityLayer}
	\item \textbf{SAP-MI: } Als \ac{SAP} zu dem Cross Layer Management Layer \label{architektur_managementLayer}
\end{itemize}
\todo{ISO 21217 finden - Scheinbar infos über die Layer}

\todo{mal in ETSI TS 102 723-10 reinsehen, ob da was interessantes zu den SAP drin steht}

Der Access Layer ist nicht auf ein bestimmtes Übertragungsprotokoll festgelegt. Beispiele für ein Übertragungsprotokoll sind ITS-G5, WiFi, BlueTooth, Ethernet\dots In einer reinen \ac{C2C} Kommunikation bietet sich aber vor allem ITS-G5 an, für eine allgemeine \ac{ITS} Verbindung haben die anderen Übertragungsprotokolle aber auch ihre Berechtigung. Diese Übertragungsprotokolle müssen aber den \ac{ITS} Protokollstack transparent übertragen. 

Der folgende Abschnitt beschreibt den Access Layer und legt G5 zugrunde.
 
\begin{figure}
	%\includegraphics[width=0.75\textwidth]{content/images/02_architektur/accessLayer.pdf}
	\missingfigure{Access Layer Bild noch aus dem Standard 302 665 herauskopieren }
	\caption{Darstellung des ITS G5 Access Layers \cite{etsi302663}}
	\label{fig:architektur_accessLayer}
\end{figure}

Die Grafik \ref{fig:architektur_accessLayer} entspricht des untersten Layer der Grafik \ref{fig:funktionsweise_referenceArchitecture}. In der Grafik ist zu erkennen, dass der Access Layer zwei \ac{SAP} besitzt.


Im Access Layer finden eine Periodisierung und eine Aufteilung des Datenverkehrs in Logic Channels statt. Diese Aufgaben werden mit verschiedenen Ansätzen gelöst. Deswegen sind sie keinem Sublayer genau zuzuordnen, sondern müssen in der Beschreibung der Sublayer gesondert betrachtet werden.


Eine Funktion des Access Layers ist das im Abschnitt \ref{architektur_dcc} beschriebene \ac{DCC}. Hier finden die Mechanismen Transmit Power Control (TPC), \ac{DCC} sensitivity control (DSC), Transit rate control (TRC), transmit datarate control (TDC) und DCC access control (TAC) statt. TPC regelt die Auslastung der Kanäle. Dazu werden Grenzen definiert, die TPC überwacht und einhält. TRC überwacht die Zeiten von Datenpaketen. Dazu gehören beispielsweise die Latenz eines Pakets aber auch die Intervalle zwischen Paketen. TDC überwacht die reine Datenrate eines Channels. Dabei wird nicht nur die maximale Datenrate überwacht, es wird auch beispielsweise die minimale Datenrate überwacht. DSC überwacht, ob der Sender bereit zum Senden ist. Dazu wird anhand von definierten Grenzwerten gemessen, ob der Sender am Senden ist oder nicht. TAC regelt den Kanalzugriff. \todo{Kanalzugriff erklären, wenn ich weiß, was Kanäle sind. Quelle für TAC: \cite{etsi102687}}

\subsubsection{Physical Layer (PHY)}
Der Physical Sublayer verbindet physikalisch zu dem Kommunikationsmedium.

\subsubsection{Data Link Layer (DLL)}
Der Data Link Sublayer kann wiederum in den Medium Access Control (MAC) Sublayer und den Logical Link Control (LLC) Sublayer aufgeteilt werden. Der MAC Sublayer regelt den Zugriff auf das Kommunikationmeduim.


\ac{ITS} bietet die Funktionalität von logischen Kanälen. 




\subsection{Networking \& Transporting \label{architektur_networkingTransporting}}
Der Networking \& Transporting Layer enthält mehrere verschiedene Netzwerk und Transport Protokolle und entspricht den \ac{OSI} Layern 3 und 4. Die Aufgabe ist das Routing und der Ende zu Ende Transport von Daten. Er wird im Kapitel \ref{chap:networklayer} genauer beschrieben. 


\subsection{Facilities}
Der Facilities Layer entspricht den \ac{OSI} Layern 5, 6 und 7. Er bietet eine Sammlung von Funktionen, die die \ac{ITS} Anwendungen unterstützen. Der Layer bietet Datenstrukturen um verschiedene Date zu speichern, zu sammeln und zu verwalten. Er wird im Kapitel \ref{chap:facilitylayer} genauer erklärt.

\subsection{Applications}
Im Applications Layer werden die Use Cases realisiert. Ihnen steht der \ac{ITS} Protokoll Stack zu Verfügung. Eine genauere Beschreibung des Application Layers findet im Kapitel \ref{chap:applicationlayer} statt.

\section{Cross/Vertical Layer}
Die Cross Layer weichen stark vom \ac{OSI} Modell ab. Sie erweitern die traditionellen Layer, die jeweils nur ein Interface zum nächst höheren, bzw. tieferen Layer haben um Layer, die Interfaces zu allen anderen Layern haben. Durch die Interfaces zu allen Layern ergeben sich neue Möglichkeiten. So kann beispielsweise im Application Layer die genutzte Bandbreite an die im Physical Layer zur Verfügung stehende Bandbreite angepasst werden. Dadurch werden Überlastungen, die sich auf die Latenz auswirken oder zu fehlerhaften Übertragungen führen bereits im im Vorfeld vermieden.

\subsection{Management Layer \label{architektur_managementLayer}}
Der Management Layer übernimmt Alle Aufgaben, die mit der Verwaltung einer \ac{ITS} Station und deren Protokollstack zusammenzufassen sind. Vereinfacht gesagt verwaltet er im Protokollstack die Cross Layer Funktionalität. 


\begin{figure}
	\includegraphics[width=0.75\textwidth]{content/images/02_architektur/managementLayer.pdf}
	\caption{Der Management Layer im Überblick \cite{etsi2010302}}
	\label{fig:architektur_managementLayer}
\end{figure}

In der  \autoref{fig:architektur_managementLayer} ist der Management Layer mit seinen Interfaces und Untereinheiten dargestellt. Er hat zu jedem anderen Layer ein Interface. Die fünf Untereinheiten ergeben sich aus den definierten Funktionalitäten des Management Layers. Die folgende Auflistung der Funktionalitäten ist dem Standard \cite{etsi2010302} entnommen:

\begin{itemize}
	\item Cross-interface Management
	\item Kommunikation zwischen Einheiten gem. ETSI TS 102 723-1
	\item Netzwerkmanagement
	\item Kommunikationsservice Management
	\item \ac{ITS} Anwendungs Management
	\item Station Management
	\item Management der allgemeinen Congestion Control
	\item Management des Service Advertisement
	\item Management des Systemschutzes 
	\item Eine alleine Informationsbasis
	\item Die Möglichkeit die verschiedenen Layer zu verbinden
\end{itemize}

\subsubsection{ITS Service Advertisement}
\ac{ITS} Service Adverticement ist der Mechanismus, mit dem eine \ac{ITS} Station \ac{ITS} Services erkennen kann. Bei diesem Mechanismus macht eine \ac{ITS} Station, in dem Fall der Service Provider, aktiv ihre Services anderen \ac{ITS} Stations, in dem Fall Service User, bekannt. Eine Möglichkeit, die Services bekannt zu machen ist das FAST Service Advertisement. Es ist im Standard ISO/IEC 24102 definiert und eignet sich für die Luftschnittstelle mit lediglich einem Hop. Beim FAST Service Advertisement wird ein Advertisement Manager benötigt. Dieser empfängt die Service Adverticements von den anderen Service Providern und sendet die Service Adverticements der eigenen \ac{ITS} Station in regelmäßigen Abständen aus.

Für das Aussenden von Service Advertisements gibt es \ac{SAM}. \autoref{architektur_darstellungSAMHeader} zeigt den Aufbau einer \ac{SAM}. Sie besteht aus einem Header und einem Body. Der Header enthält die Elemente:

\begin{itemize}
	\item samID: Identifiziert die \ac{SAM}
	\item Version: Die Versionsnummer der \ac{SAM}
	\item stationID: Die ID des sendenden Service Providers
\end{itemize}

Der Body enthält die folgenden Elemente:
\begin{itemize}
	\item serviceList: Eine Liste mit den angebotenen Services. Sie sind nach dem Standard ISO 17419 eindeutig kodiert
	\item channelList:  Eine Information, welche Channels für die Service Operation Phase genutzt werden
	\item ipServList: Informationen über Services, die angeboten wurden und der Service Operation Phase IPv6 benötigen.
\end{itemize} 


\begin{figure}[h]
	\begin{bytefield}{40}
		\wordbox{1}{Service Advertisement Message SAM} \\
		\bitbox{16}{Header} & \bitbox{24}{Body} \\
		\bitbox{4}{samID} & \bitbox{4}{Version} & \bitbox{8}{stationID} & \bitbox{8}{serviceList} & \bitbox{8}{channelList} & \bitbox{8}{ipServList}
		\end{bytefield}
	\caption{Darstellung eines SAM Pakets}
	\label{architektur_darstellungSAMHeader}
\end{figure}

Der Service User beantwortet die \ac{SAM} mit einer \ac{CTX}. Die \ac{CTX} ist ähnlich aufgebaut wie die \ac{SAM}. In der \autoref{architektur_darstellungCTXHeader} ist eine \ac{CTX} dargestellt.

\begin{figure}[h]
	\begin{bytefield}{32}
		\wordbox{1}{Context Message CTX} \\
		\bitbox{16}{Header} & \bitbox{16}{Body} \\
		\bitbox{4}{ctxID} & \bitbox{4}{Version} & \bitbox{8}{clientID} & \bitbox{8}{servContext\-List} & \bitbox{8}{ipContext\-List}
		\end{bytefield}
	\caption{Darstellung eines CTX Pakets}
	\label{architektur_darstellungCTXHeader}
\end{figure}

Der Header der \ac{CTX} entspricht dem einer \ac{SAM}. Hier wird aber anstatt der \ac{ID} des Providers die \ac{ID} des Clients mitgesendet. Im Body unterscheiden sich die Nachrichten. 

Die Body Inhalte einer \ac{CTX}:
\begin{itemize}
	\item servContextList: Informationen über den Service Kontext, der beim Service User verfügbar ist. Kann als Antwort auf einen angebotenen Service in der serviceList der \ac{SAM} vorliegen.
	\item ipContextList: Informationen über Service Kontexte, die beim Service User verfügbar sind und IPv6 benötigen. Kann als Antwort auf einen Service, der in der ipServList der \ac{SAM} angeboten wurde vorliegen.
\end{itemize}  

Das Bekanntmachen von Services kann auf zwei Arten erfolgen. Die Möglichkeiten unterscheiden sich darin, dass bei der ersten Möglichkeit die \ac{SAM} vom Service User mit einer \ac{CTX} beantwortet wird. Bei der zweiten Möglichkeit wird die \ac{SAM} nicht beantwortet. Grundsätzlich laufen die Möglichkeiten aber gleich ab.  

Die Kommunikation zwischen User und Provider kann man in zwei Phasen aufteilen. Die Service Initialization Phase und die Service Operation Phase.

Der Zweck der Service Invitation Phase ist es die Session aufzubauen. Dabei wird der Service User mit einer \ac{SAM} eingeladen. Während der Service Invitation Phase wird zwischen den beschriebenen Möglichkeiten unterschieden. Ob eine \ac{SAM} von einer \ac{CTX} bestätigt wird, hängt davon ab, ob es sich beim Service User um eine \ac{ITS} application class oder eine \ac{ITS} application handelt. Der Unterschied zwischen \ac{ITS} Application Class und \ac{ITS} Application ist, dass von einem Application Objekt mehrere Kontexte existieren können. Jeder Kontext kann auf eine \ac{ITS} Application referenziert werden. Bei der Übertragung wird der Unterschied durch  den \ac{ASN.1} Typ \glqq DSRCapplicationEntityID\grqq~als Markierung deutlich gemacht. 

Bei der Einladung von Application Classes wird die \ac{SAM} durch eine \ac{CTX} bestätigt. Bei Applications wird keine \ac{CTX} versendet. Die Service Invitation Phase wird als erfolgreich angesehen, sobald das erste \glqq REQUW\grqq~oder \glqq REQN\grqq~versendet wird. 

Nach der erfolgreichen Service Invitation Phase folgt die Service Operation Phase. \todo{rausfinden, was in dieser Phase statt findet}
\begin{figure} 
  \centering 
   \subfigure[Ohne Bestätigung durch CTX] {\includegraphics[width=0.45\textwidth]{content/images/02_architektur/serviceAdvertisementInit.pdf}}\qquad 
   \subfigure[Mit Bestätigung durch CTX]{ \includegraphics[width=0.45\textwidth]{content/images/02_architektur/serviceAdvertisementInitCTX.pdf}}
  \caption{Ablauf der Phasen des Fast Service Advertisement Protocol \cite{iso24102-5}} 
  \label{fig:architektur_ablaufPhasen}
\end{figure}
\todo{Prüfen, warum die Pfeile bei den Verbindungen genau umgekehrt sind}

In \autoref{fig:architektur_ablaufPhasen} wird der Ablauf der Phasen darstellt. Die einzelnen Schritte der Kommunikation bedeuten ausgeschrieben:
\begin{itemize}
	\item Request with no response expected (REQN)
	\item Request with response expected (REQW)
	\item Response to a request (RES)
\end{itemize}

Beschrieben in \cite{etsi102723-2}

\todo{Management Layer genauer beschreiben}

\todo{DCC genauer erklären und Text anpassen}

\subsubsection{Decentralizied Congestion Control\label{architektur_dcc}}
Congestion lässt sich aus dem Englischen mit Stau übersetzen. \ac{DCC} ist ein Mechanismus, der verhindern soll, dass Staus auftreten. Besonders bei \ac{ITS} Anwendungen kommt es auf zuverlässige und Übertragungswege an. Es werden hohe Anforderungen an die Verfügbarkeit und die Latenzen der Übertragungen gestellt. An Luftschnittstellen sind diese Anforderungen ohne eine Komponente wie \ac{DCC} kaum zu erfüllen. eine Der Standard \cite{etsi102687} definiert folgende Anforderungen an \ac{DCC}:
\begin{itemize}
	\item Eine faire Verteilung von Ressourcen und ein fairer Kanalzugriff zwischen allen \ac{ITS} Stationen in der gleichen Kommunikationszone
	\item Die Auslastung der Kanäle muss unter vordefinierten Werten bleiben. Dies muss durch eine periodische Messung sicher gestellt werden
	\item Reservierung von Kommunikationsressourcen für das Verbreiten von hoch priorisieren ereignisgesteuerten Nachrichten
	\item Schnelle Übernahme einer wechselnden Umgebung (busy / free radio channel)
	\item Die Änderungen in den Kontrollschleifen müssen in den definierten Grenzen bleiben
	\item Es muss den spezifischen Systemanforderungen, beispielsweise Zuverlässigkeit, entsprechen
\end{itemize}

Aus diesen Anforderungen lässt sich herauslesen, dass das Vermeiden von Staus durch mehrere Mechanismen realisiert wird. Eine wichtige Eigenschaft von \ac{DCC} ist, dass es im Management Layer angesiedelt ist. Diese Tatsache ermöglicht es \ac{DCC} seine Aufgaben parallel in mehreren Layern zu realisieren. Die \ref{fig:architektur_dccArchitektur} zeigt die Architektur von \ac{DCC}. Die Abbildung zeigt den \ac{ITS} Protokoll Stack in den die \ac{DCC} Komponenten und Interfaces eingezeichnet sind. Der Vorteil dass die Layer vernetzt sind ist, dass der Stau wirklich vermieden werden kann und nicht nur die Auswirkungen des Staus behandelt werden müssen. Ein Beispiel dafür ist, dass \ac{DCC} das Trafficaufkommen bereits im Network Layer an das Medium anpassen und einzelne Dienste priorisieren kann. IEEE 802.11 beispielsweise muss bei einer Überlast, bzw. einem Pufferüberlauf, Frames verwerfen. Dieses Verwerfen muss durch Protokolle höherer Layer abgefangen werden und führt aufgrund von Retransmissions zu höheren Latenzen und einer ingesamt höheren Netzwerkauslastung. 
\todo{Stimmt das mit dem Verwerfen eigentlich?}

\begin{figure}
	\includegraphics[width=0.75\textwidth]{content/images/02_architektur/dccArchitektur.pdf}
	\caption{Die Architektur von DCC \cite{etsi102687}}
	\label{fig:architektur_dccArchitektur}
\end{figure}

Für die Kommunikation mit mehreren Layern sind in der  \ac{DCC} Architektur vier Komponenten definiert. Die Komponenten sind mit den \ac{DCC} Interfaces verbunden, die selber auf den Interfaces der Layer zugeordnet werden. Die Komponenten werden in den Layern erklärt, in denen sie liegen. 

Die DCCmgmt Komponente ist im Management Layer angeordnet. \todo{In jeden Layer was zu DCC schreiben}



\subsection{Security Layer \label{architektur_securityLayer}}


\section{Data Security}

\section{Verwendete Protokolle}




