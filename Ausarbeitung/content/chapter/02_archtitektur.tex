\chapter{Architektur  \label{chap_archtitektur}}
Die Netzwerkarchitektur von \ac{ITS} umfasst sowohl interne als auch externe Netzwerke. Laut Standard \cite{etsi302636-3}, bzw. Standard \cite{etsi102636-3} sind dabei folgende externe Netzwerke erfasst:

\begin{itemize}
 	\item ITS ad hoc network.
	\item Access network (ITS access network, public access network, private access network).
	\item Core network (e.g. the Internet).
\end{itemize}

Auf der Grafik \autoref{fig:architektur_ueberblickNetzwerke} sind die verschiedenen Netzwerke visualisiert. Diese Art der Darstellung entspricht der höchsten Abstraktionsebene. Die verschiedenen Netzwerke sind in der Grafik als Wolken dargestellt. Neben den Netzwerken sind auch die Verbindungen visualisiert.


Zusätzlich zu den hier beschriebenen Netzwerken kann eine \ac{ITS-S} ein eigenes Netzwerk, das die Teilkomponenten der \ac{ITS-S} verbindet, betreiben. Die verschiedenen Netzwerke werden benötigt, damit alle Dienste mit ihren verschiedenen Anforderungen bedient werden können.

\begin{figure}[h]
	\includegraphics[width=0.99\textwidth]{content/images/02_architektur/uebersichtExterneNetzwerke.pdf}
	\caption{Überblick über die externen Netzwerke \cite{etsi302636-3}}
	\label{fig:architektur_ueberblickNetzwerke}
\end{figure}

\section{Übersicht über die verschiedenen Netzwerke \label{architektur_ueberblickNetzwerke}}
Dieser Abschnitt soll lediglich eine Übersicht über die verwendeten Netzwerke geben. Eine weitere Erklärung der Netzwerke findet an dieser Stelle nicht statt und ist nicht Gegenstand dieser Ausarbeitung. Die Netzwerke dürfen auch nicht isoliert betrachtet werden. Im Abschnitt \autoref{funktionsweise_funktionaleKomponenten} wurden funktionale Komponenten mit Routingfunktionalitäten vorgestellt. Diese können die Netze und somit ihre Vorteile, bzw. ihre Dienste, miteinander verbinden.

Selbstverständlich benötigen die \ac{ITS-S} Zugang zu einem der im Folgenden aufgeführten Netze. Der Zugang zum Core Network \autoref{achitektur_coreNetwork} erfolgt über eins der anderen Netze. 

\subsection{ITS Ad Hoc Network\label{achitektur_adHocNetwork}}
Das \ac{ITS} Ad Hoc Netzwerk ist das Netzwerk für die Kommunikation zwischen \ac{IRS}, \ac{IVS} und \ac{PSS}. Die Kommunikation findet über die Luftschnittstelle statt. Sie ist in ihrer Reichweite begrenzt, dafür ist sie mobil einsetzbar. Die Drahtlose Kommunikation wird im Normalfall über den Standard ITS-G5 ermöglicht.


\subsection{ITS Access Network \label{architektur_itsAccessNetwork}}
ITS Access Networks werden zur Vernetzung von \ac{ITS} Komponenten verwendet. Diese Netzwerke bieten den Zugang für die entsprechenden \ac{ITS} Services. Sie werden als eigene Netzwerke realisiert. \ac{ITS-S} werden durch Access Networks verbunden. Das bedeutet, dass \ac{IRS} untereinander über Access Networks verbunden sein können, es können aber auch Stations, die normalerweise Ad Hoc miteinander kommunizieren dieses Netz nutzen. 

\subsection{Public Access Network}
Ein Public Access Network ermöglicht den Zugang in öffentlich zugängliche Mehrzwecknetzwerke. Dieses Netzwerk kann beispielsweise dazu genutzt werden, um \ac{ITS-S} mit dem Core Netzwerk zu verbinden. 

\subsection{Private Access Network}
Ein Private Access Network reguliert den Zugang durch die Teilnehmer. Die angebotenen Datendienste stehen nur einer bestimmten Gruppe von Nutzern zur Verfügung. Mit Private Access Networks besteht die Möglichkeit, eine gesicherte Verbindung in ein anderes Netzwerk aufzubauen. So kann beispielsweise ein \ac{IVS} auf das Intranet einer Firma zugreifen. 

\subsection{Core Network \label{achitektur_coreNetwork}}
Das Core Network ist ein Verbindungsnetz. Es hat keine \ac{ITS} Funktionalitäten und wird im Standard auch nicht weiter spezifiziert. Es wird in Verbindung mit den Public Access Network dazu genutzt, traditionelle Dienste, wie Internet oder Email, anzubieten.
 
\begin{figure}
	\includegraphics[width=0.95\textwidth]{content/images/02_architektur/netzwerkSzenario.pdf}
	%\vspace{2cm}
	\includegraphics[width=0.75\textwidth]{content/images/02_architektur/verbindungenNetzwerkSzenario.pdf}
	\caption{Netzwerkszenario mit dazugehöriger Implementierung \cite{etsi302636-3}}
	\label{fig:architektur_netzwerkSzenario}
\end{figure}

Die \autoref{fig:architektur_netzwerkSzenario} stammt aus dem Standard \cite{etsi302636-3}. Dort ist im oberen Teil der Grafik ein Szenario beschrieben, welche Netzwerke miteinander verbunden sein können.  Zu erkennen ist, dass die Netzwerke ITS Ad Hoc Netzwerk (\autoref{achitektur_adHocNetwork}), ITS Access Network  (\autoref{architektur_itsAccessNetwork}) und das Core Network (\autoref{achitektur_coreNetwork}) miteinander verbunden sein sollen.

Der untere Teil der Grafik zeigt eine Implementierungsmöglichkeit dieses Szenarios. Die hellen Rechtecke beschreiben die Komponenten, die in dieser Implementierung im System integriert sind, das dunkle Rechteck beschreibt die funktionale Komponente, die in diesem System beteiligt ist. Die Linien sind mit dem Typ des Netzwerks, welches sie repräsentieren, beschriftet und zusätzlich mit dem Network Reference Point, den sie benutzen, beschriftet.

Auflistung und kurze Beschreibung der genutzten Network Reference Points:
\begin{itemize}
	\item \textbf{RA: } Reference Point zwischen \ac{IVS} über das ITS Ad Hoc Network
	\item \textbf{RB: } Reference Point zwischen \ac{IVS} und \ac{IRS} über das ITS Ad Hoc Network
	\item \textbf{RC: } Reference Point zwischen \ac{PSS} über das ITS Ad Hoc Network
	\item \textbf{RD: } Reference Point zwischen \ac{PSS} und \ac{IRS} über das ITS Ad Hoc Network
	\item \textbf{RE: } Reference Point zwischen \ac{IVS} und \ac{PSS} über das ITS Ad Hoc Network
	\item \textbf{RF: } Reference Point zwischen \ac{IRS} über das ITS Access Network
	\item \textbf{RG: } Reference Point zwischen \ac{IRS} und einem ITS-S Border Router \footnote{Der Border Router muss nicht explizit aufgeführt werden, da er als funktionale Komponente Teil einer Komponente ist. \label{ftn:borderRouter}} über das ITS Access Network 
	\item \textbf{RH: } Reference Point zwischen \ac{ICS} und ITS-S Border Router \footref{ftn:borderRouter} über das Core Network		
\end{itemize}

Erkennbar ist in dieser Implementierung, dass sich für mobile Stations Ad Hoc Netzwerke verwendet wurden. Diese haben den Vorteil, dass sie bereits in der Spezifikation mit der Luftschnittstelle ITS-G5 ausgestattet sind, was eine Mobilität erst ermöglicht. Was auch erkennbar ist, ist, dass die reinen ITS Netzwerke durch einen Border Router vom Core Network getrennt sind. Auch wenn hier nicht explizit aufgeführt, die \ac{ICS} benötigt in diesem Fall auch einen Border Router.

 
\section{ITS Station Reference Architecture}
Eine Referenzarchitektur beschreibt ein allgemeines Modell einer Architektur. Modelle können untereinander verglichen werden. Eine Architektur ist abstrakt beschrieben. Durch die Beschreibung einer Architektur können das Verhalten und die Anforderungen an das System beschrieben werden. Die Implementierung der Funktionalitäten ist nicht vorgegeben. 

Die \ac{ITS-S} Reference Architecture unterscheidet sich grundlegend von bekannten Architekturen. Da sie während der Entwicklung an das \ac{OSI} Modell angelehnt war, ergeben sich einige Parallelen:
\begin{itemize}
	\item Trennung der einzelnen Layer
	\item Definition von Service Primitiven zwischen den Layern
	\item Die Standards beziehen die Layer auf die \ac{OSI} Layer. 
\end{itemize}

Der direkte Vergleich mit dem \ac{OSI} Modell und die Zuordnung der Layer wird in \autoref{fig:architektur_vergleichItsOsi} deutlich. Diese Grafik enthält sowohl die Layer von \ac{ITS} als auch die Layer des \ac{OSI} Modells. In der Grafik sind  vier Säulen enthalten. Von links nach rechts beschrieben stellen die Säulen den Management Layer  (\autoref{architektur_managementLayer}), die \ac{ITS} horizontal Layer (\autoref{architektur_horzontalLayer}), die \ac{OSI} Layer und den Security Layer (\autoref{architektur_securityLayer}) dar. Die \ac{ITS} Layer werden im Folgenden genauer beschrieben. 

\begin{figure}[h]
	\includegraphics[width=0.99\textwidth]{content/images/02_architektur/vergleichITS-OSI.pdf}
	\caption{Der Vergleich zwischen ITS und OSI \cite{ts102940}}
	\label{fig:architektur_vergleichItsOsi}
\end{figure}

Obwohl das \ac{ITS-S} Reference Protocol bei der Entwicklung an das \ac{OSI} Modell angelehnt wurde gibt es jedoch einen gravierenden Unterschied: In der \ac{ITS-S} Reference Architecture sind Cross Layer vorgesehen. Das \ac{OSI} Referenzmodell ist wasserfallartig aufgebaut. Das bedeutet, dass die einzelnen Layer übereinander angeordnet sind. Jeder Layer hat jeweils nur zu dem direkt darüber- und darunterliegenden Layer eine Schnittstelle. Cross Layer sind Layer, die in mehrere dieser Schichten Schnittstellen haben. Sie erweitern die vorhanden Layer in horizontaler Richtung. Im Fall der \ac{ITS} Reference Architecture sind das die Layer \glqq Management\grqq~ und \glqq Security\grqq. Sie haben Schnittstellen, bzw. Primitiven in alle anderen Layer. 

\begin{figure}
	\includegraphics[width=0.75\textwidth]{content/images/02_architektur/stationReferenceArchitecture.pdf}
	\caption{Darstellung der ITS Station Reference Architecture \cite{ts102940}}
	\label{fig:funktionsweise_referenceArchitecture}
\end{figure}

\section{Horizontal Layer \label{architektur_horzontalLayer}}
Dieser Abschnitt beschreibt die Layer, die klassisch übereinander angeordnet sind. Die Layer und ihre Funktionen entsprechen im Groben den Layern des \ac{OSI} Modells. Ihre Aufteilung und Aufgabenverteilung unterscheidet sich aber. Sofern eine direkte Zuordnung zwischen \ac{ITS} und \ac{OSI} treffbar ist, wird diese gemacht.

\subsection{Access Layer \label{subsec:accessLayer}}
Der Access Layer von ITS entspricht den \ac{OSI} Layern 1 und 2. Er besteht aus zwei Sublayern und hat drei Interfaces, bzw. \ac{SAP}. Die Interfaces dienen der Kommunikation mit den anderen Layern.

\begin{itemize}
	\item \textbf{SAP-IN: } Als \ac{SAP} zu dem nächst höheren Layer  Networking \& Transporting (\autoref{architektur_networkingTransporting})
	\item \textbf{SAP-SI: } Als \ac{SAP} zu dem Cross Layer Security Layer (\autoref{architektur_securityLayer})
	\item \textbf{SAP-MI: } Als \ac{SAP} zu dem Cross Layer Management Layer (\autoref{architektur_managementLayer})
\end{itemize}

Der Access Layer übernimmt die Übertagung der Daten, ist aber nicht auf ein bestimmtes Übertragungsprotokoll festgelegt. Beispiele für Übertragungsprotokolle sind ITS-G5, WiFi, BlueTooth, Ethernet, \dots In \ac{C2C} Anwendungen kommen Dedicated Short Range Communication Protokolle zum Einsatz. Diese sind in unterschiedlichen Ländern, bzw. Kontinenten anders genormt und haben andere Namen. Im europäischen Raum wird \ac{ITS-G5} eingesetzt. \ac{ITS-G5} ist vom \ac{IEEE} als \ac{IEEE} 802.11 p genormt. Dabei handelt sich um einen Funkstandard, der im Bereich von 5 GHz arbeitet. Eine genaue Beschreibung dieses Funkstandards findet sich in \cite{en302663} wieder und würde an dieser Stelle zu weit führen. \ac{ITS-G5} unterscheidet sich von den klassischen 802.11 Standards. Ein Vergleich ist in der \autoref{fig:architektur_vergleich-G5-802} dargestellt. 

\begin{figure}
	\includegraphics[width=0.95\textwidth]{content/images/02_architektur/unterschiede-G5-Wlan.pdf}
	\caption{Vergleich zwischen G5 und klassischen 802.11 Standards \cite{g5-wlan-hardware}}
	\label{fig:architektur_vergleich-G5-802}
\end{figure}

\subsubsection{Data Link Layer (DLL)}
Der Data Link Layer besteht wiederum aus zwei Sublayern. Sie übernehmen auch Funktionalitäten von \ac{DCC}. \ac{DCC} ist im \autoref{architektur_dcc} genauer beschrieben. 
\paragraph{Medium Access Control (MAC)}
Das Medium Access Control steuert, wer übertragen darf. Diese Berechtigung wird in Abhängigkeit von der Belegung der Schnittstelle vergeben. Bekannte Protokolle für die Zugriffskontrolle auf das Medium sind beispielsweise Token-Ring oder CSMA/CD. 

In \ac{ITS-G5} wird das Verfahren \ac{EDCA} verwendet. Es baut auf einem CSMA/CA Verfahren auf, erweitert die klassischen Algorithmen aber um \ac{QOS} Elemente. 

\paragraph{Logical Link Control (LLC)}
Der Logic Link Control ist im Standard ANSI/IEEE Std. 802.2 definiert. Dieser ist  proprietär und lag uns nicht vor.

\subsubsection{Physical Layer (PHY)}
Der Physical Layer verbindet physikalisch mit dem Übertragungsmedium. Prinzipiell kann \ac{ITS} mit fast jedem Medium durchgeführt werden, für die mobilen Anwendungen bietet sich aber eine Übertragung über die Luftschnittstelle besonders an. Mit  \ac{ITS-G5} wurde ein besonders für dieses Zweck geeigneter Standard entworfen. Im PHY werden die physikalischen Channels verwaltet. Ein physikalischer Channel entspricht einem Frequenzbereich. 

Im Fall von \ac{ITS-G5} teilt der Physical Layer die gesamte Bandbreite in 52 Unterkanäle (subcarriers) ein. Sie sind alle in der Bandbreite beschränkt. Der Physical Layer regelt weiterhin das verwendete Modulationsverfahren und die Übertragungsrate. Gemäß \cite{en302663} muss er mindestens die Transferraten 3 Mbit/s, 6 Mbit/s, und 12 Mbit/s beherrschen. 
 
\begin{figure}
	\includegraphics[width=0.95\textwidth]{content/images/02_architektur/accessLayer.pdf}
	%\missingfigure{Access Layer Bild noch aus dem Standard 302 665 herauskopieren }
	\caption{Darstellung des ITS G5 Access Layers \cite{en302665}}
	\label{fig:architektur_accessLayer}
\end{figure}


\subsection{Networking \& Transporting \label{architektur_networkingTransporting}}
Der Networking \& Transporting Layer enthält mehrere verschiedene Netzwerk- und Transportprotokolle und entspricht den \ac{OSI} Layern 3 und 4. Die Aufgabe ist das Routing und der Ende zu Ende Transport von Daten. Er wird im \autoref{chap:networklayer} genauer beschrieben. 


\subsection{Facilities \label{architektur_facilitiesLayer}}
Der Facilities Layer entspricht den \ac{OSI} Layern 5, 6 und 7. Er bietet eine Sammlung von Funktionen, die die \ac{ITS} Anwendungen unterstützen. Der Layer bietet Datenstrukturen um verschiedene Daten zu speichern, zu sammeln und zu verwalten. Er wird im \autoref{chap:facilitylayer} genauer erklärt.

\subsection{Applications}
Im Applications Layer werden die Use Cases realisiert. Dafür bietet der Application Layer eine Möglichkeit, Application auszuführen. Den Applications steht der gesamte \ac{ITS} Protokoll Stack zur Verfügung. Über Interfaces können die Elemente des Protokoll Stacks angesprochen werden.
Eine weitere Beschreibung des Application Layers und eine Auswahl seiner Use Cases findet im \autoref{sec:applicationLayer} statt.

\section{Cross/Vertical Layer}
Die Cross Layer weichen stark vom \ac{OSI} Modell ab. Sie erweitern die traditionellen Layer, die jeweils nur ein Interface zum nächst höheren, bzw. tieferen Layer haben um Layer, die Interfaces zu allen anderen Layern haben. Durch die Interfaces zu allen Layern ergeben sich neue Möglichkeiten. Ein Konzept kann mithilfe der Cross Layer im gesamten \ac{ITS} Protokollstack umgesetzt werden. So kann beispielsweise im Application Layer die genutzte Bandbreite an die im Physical Layer zur Verfügung stehende Bandbreite angepasst werden. Dadurch werden Überlastungen, die sich auf die Latenz auswirken oder zu fehlerhaften Übertragungen führen bereits im im Vorfeld vermieden. \ac{DCC} kann somit Staus wirkungsvoll vermeiden, anstatt nur die Symptome zu mindern.

\subsection{Management Layer \label{architektur_managementLayer}}
Der Management Layer übernimmt alle Aufgaben, die mit der Verwaltung einer \ac{ITS-S} und deren Protokollstack zusammenzufassen sind. Vereinfacht gesagt, verwaltet er im Protokollstack die Cross Layer Funktionalität. 


\begin{figure}
	\includegraphics[width=0.75\textwidth]{content/images/02_architektur/managementLayer.pdf}
	\caption{Der Management Layer im Überblick \cite{en302665}}
	\label{fig:architektur_managementLayer}
\end{figure}

In der  \autoref{fig:architektur_managementLayer} ist der Management Layer mit seinen Interfaces und Untereinheiten dargestellt. Er hat zu jedem anderen Layer ein Interface. Die fünf Untereinheiten ergeben sich aus den definierten Funktionalitäten des Management Layers. In der aktuellen Version der Standards sind die Untereinheiten der Layers noch nicht exakt definiert. Die folgende Auflistung der Funktionalitäten ist dem Standard \cite{en302665} entnommen.

\begin{itemize}
	\item Cross-interface Management
	\item Kommunikation zwischen Einheiten gem. ETSI TS 102 723-1
	\item Netzwerkmanagement
	\item Kommunikationsservice Management
	\item \ac{ITS} Anwendungs Management
	\item Station Management
	\item Management der allgemeinen Congestion Control
	\item Management des Service Advertisement
	\item Management des Systemschutzes 
	\item Eine alleine Informationsbasis
	\item Die Möglichkeit die verschiedenen Layer zu verbinden
\end{itemize}

\subsubsection{Management Information Base}
Die Management Information Base verwaltet die Dienste, die im Netzwerk angeboten werden. \ac{ITS} verwendet das Simple Network Management Protocol in der Version 2. Mit diesem Protokoll können Komponenten im Netzwerk zentral überwacht und gesteuert werden. Da die Komponenten unterschiedliche Services anbieten, werden die Komponenten mit ihren angebotenen Services in der Management Information Base gespeichert.
Laut Standard \cite{ts102723-2} wird es für \ac{ITS} keine besondere Management Information Base geben, es werden die aktuell im Simple Network Management Protocol verfügbaren \ac{ASN.1} Elemente genutzt. 


\subsubsection{ITS Service Advertisement}
\ac{ITS} Service Adverticement ist der Mechanismus, mit dem eine \ac{ITS-S} \ac{ITS} Services erkennen kann. Bei diesem Mechanismus macht eine \ac{ITS-S}, in dem Fall der Service Provider, aktiv ihre Services anderen \ac{ITS-S}, in dem Fall Service User, bekannt. Eine Möglichkeit, die Services bekannt zu machen ist das FAST Service Advertisement. Es ist im Standard ISO/IEC 24102 definiert und eignet sich für die Luftschnittstelle mit lediglich einem Hop. Beim FAST Service Advertisement wird ein Advertisement Manager benötigt. Dieser empfängt die Service Adverticements von den anderen Service Providern und sendet die Service Adverticements der eigenen \ac{ITS-S} in regelmäßigen Abständen aus.

Für das Aussenden von Service Advertisements gibt es \ac{SAM}. \autoref{architektur_darstellungSAMHeader} zeigt den Aufbau einer \ac{SAM}. Sie besteht aus einem Header und einem Body. Der Header enthält die Elemente:

\begin{itemize}
	\item samID: Identifiziert die \ac{SAM}
	\item Version: Die Versionsnummer der \ac{SAM}
	\item stationID: Die ID des sendenden Service Providers
\end{itemize}

Der Body enthält die folgenden Elemente:
\begin{itemize}
	\item serviceList: Eine Liste mit den angebotenen Services. Sie sind nach dem Standard ISO 17419 eindeutig kodiert
	\item channelList:  Eine Information, welche Channels für die Service Operation Phase genutzt werden
	\item ipServList: Informationen über Services, die angeboten wurden und der Service Operation Phase IPv6 benötigen.
\end{itemize} 


\begin{figure}[h]
	\begin{bytefield}{40}
		\wordbox{1}{Service Advertisement Message SAM} \\
		\bitbox{16}{Header} & \bitbox{24}{Body} \\
		\bitbox{4}{samID} & \bitbox{4}{Version} & \bitbox{8}{stationID} & \bitbox{8}{serviceList} & \bitbox{8}{channelList} & \bitbox{8}{ipServList}
		\end{bytefield}
	\caption{Darstellung eines SAM Pakets  \cite{iso24102-5}}
	\label{architektur_darstellungSAMHeader}
\end{figure}

Der Service User beantwortet die \ac{SAM} mit einer \ac{CTX}. Die \ac{CTX} ist ähnlich aufgebaut wie die \ac{SAM}. In der \autoref{architektur_darstellungCTXHeader} ist eine \ac{CTX} dargestellt.

\begin{figure}[h]
	\begin{bytefield}{32}
		\wordbox{1}{Context Message CTX} \\
		\bitbox{16}{Header} & \bitbox{16}{Body} \\
		\bitbox{4}{ctxID} & \bitbox{4}{Version} & \bitbox{8}{clientID} & \bitbox{8}{servContext\-List} & \bitbox{8}{ipContext\-List}
		\end{bytefield}
	\caption{Darstellung eines CTX Pakets  \cite{iso24102-5}}
	\label{architektur_darstellungCTXHeader}
\end{figure}

Der Header der \ac{CTX} entspricht dem einer \ac{SAM}. Hier wird aber anstatt der \ac{ID} des Providers die \ac{ID} des Clients mitgesendet. Im Body unterscheiden sich die Nachrichten. 

Die Body Inhalte einer \ac{CTX}:
\begin{itemize}
	\item servContextList: Informationen über den Service Kontext, der beim Service User verfügbar ist. Kann als Antwort auf einen angebotenen Service in der serviceList der \ac{SAM} vorliegen.
	\item ipContextList: Informationen über Service Kontexte, die beim Service User verfügbar sind und IPv6 benötigen. Kann als Antwort auf einen Service, der in der ipServList der \ac{SAM} angeboten wurde vorliegen.
\end{itemize}  

Das Bekanntmachen von Services kann auf zwei Arten erfolgen. Die Möglichkeiten unterscheiden sich darin, dass bei der ersten Möglichkeit die \ac{SAM} vom Service User mit einer \ac{CTX} beantwortet wird. Bei der zweiten Möglichkeit wird die \ac{SAM} nicht beantwortet. Grundsätzlich laufen die Möglichkeiten aber gleich ab.  

Die Kommunikation zwischen User und Provider kann man in zwei Phasen aufteilen. Die Service Initialization Phase und die Service Operation Phase.

Der Zweck der Service Invitation Phase ist es, die Session aufzubauen. Dabei wird der Service User mit einer \ac{SAM} eingeladen. Während der Service Invitation Phase wird zwischen den beschriebenen Möglichkeiten unterschieden. Ob eine \ac{SAM} von einer \ac{CTX} bestätigt wird, hängt davon ab, ob es sich beim Service User um eine \ac{ITS} application Class oder eine \ac{ITS} application handelt. Der Unterschied zwischen \ac{ITS} Application Class und \ac{ITS} Application ist, dass von einem Application Objekt mehrere Kontexte existieren können. Jeder Kontext kann auf eine \ac{ITS} Application referenziert werden. Bei der Übertragung wird der Unterschied durch  den \ac{ASN.1} Typ \glqq DSRCapplicationEntityID\grqq~als Markierung deutlich gemacht. 

Bei der Einladung von Application Classes wird die \ac{SAM} durch eine \ac{CTX} bestätigt. Bei Applications wird keine \ac{CTX} versendet. Die Service Invitation Phase wird als erfolgreich angesehen, sobald das erste \glqq REQUW\grqq~oder \glqq REQN\grqq~versendet wird. 

Nach der erfolgreichen Service Invitation Phase folgt die Service Operation Phase.

\begin{figure} 
  \centering 
   \subfigure[Ohne Bestätigung durch CTX] {\includegraphics[width=0.45\textwidth]{content/images/02_architektur/serviceAdvertisementInit.pdf}}\qquad 
   \subfigure[Mit Bestätigung durch CTX]{ \includegraphics[width=0.45\textwidth]{content/images/02_architektur/serviceAdvertisementInitCTX.pdf}}
  \caption{Ablauf der Phasen des Fast Service Advertisement Protocol \cite{iso24102-5}} 
  \label{fig:architektur_ablaufPhasen}
\end{figure}


In \autoref{fig:architektur_ablaufPhasen} wird der Ablauf der Phasen darstellt. Die einzelnen Schritte der Kommunikation bedeuten ausgeschrieben:
\begin{itemize}
	\item Request with no response expected (REQN)
	\item Request with response expected (REQW)
	\item Response to a request (RES)
\end{itemize}

Beschrieben in \cite{etsi102723-2}

\subsubsection{Decentralizied Congestion Control\label{architektur_dcc}}
Congestion lässt sich aus dem Englischen mit Stau übersetzen. \ac{DCC} ist ein Mechanismus, der verhindern soll, dass Staus auftreten. Besonders bei \ac{ITS} Anwendungen kommt es auf zuverlässige und schnelle Übertragungswege an. Es werden hohe Anforderungen an die Verfügbarkeit und die Latenzen der Übertragungen gestellt. An Luftschnittstellen sind diese Anforderungen ohne eine Komponente wie \ac{DCC} kaum zu erfüllen. Der Standard \cite{etsi102687} definiert folgende Anforderungen an \ac{DCC}:
\begin{itemize}
	\item Eine faire Verteilung von Ressourcen und ein fairer Kanalzugriff zwischen allen \ac{ITS-S} in der gleichen Kommunikationszone
	\item Die Auslastung der Kanäle muss unter vordefinierten Werten bleiben. Dies muss durch eine periodische Messung sichergestellt werden
	\item Reservierung von Kommunikationsressourcen für das Verbreiten von hoch priorisieren ereignisgesteuerten Nachrichten
	\item Schnelle Übernahme einer wechselnden Umgebung (busy / free radio channel)
	\item Die Änderungen in den Kontrollschleifen müssen in den definierten Grenzen bleiben
	\item Es muss den spezifischen Systemanforderungen, beispielsweise Zuverlässigkeit, entsprechen
\end{itemize}

Aus diesen Anforderungen lässt sich herauslesen, dass das Vermeiden von Staus durch mehrere Mechanismen realisiert wird. Eine wichtige Eigenschaft von \ac{DCC} ist, dass es im Management Layer angesiedelt ist. Diese Tatsache ermöglicht es \ac{DCC} seine Aufgaben parallel in mehreren Layern zu realisieren. Die \autoref{fig:architektur_dccArchitektur} zeigt die Architektur von \ac{DCC}. Sie zeigt den \ac{ITS} Protokoll Stack in den die \ac{DCC} Komponenten und Interfaces eingezeichnet sind. Der Vorteil dass die Layer vernetzt sind ist, dass der Stau wirklich vermieden werden kann und nicht nur die Auswirkungen des Staus behandelt werden müssen. Ein Beispiel dafür ist, dass \ac{DCC} das Trafficaufkommen bereits im Network Layer an das Medium anpassen und einzelne Dienste priorisieren kann. IEEE 802.11 beispielsweise muss bei einer Überlast, bzw. einem Pufferüberlauf, Frames verwerfen. Dieses Verwerfen muss durch Protokolle höherer Layer abgefangen werden und führt aufgrund von Retransmissions zu höheren Latenzen und einer ingesamt höheren Netzwerkauslastung. 

\begin{figure}
	\includegraphics[width=0.75\textwidth]{content/images/02_architektur/dccArchitektur.pdf}
	\caption{Die Architektur von DCC \cite{etsi102687}}
	\label{fig:architektur_dccArchitektur}
\end{figure}

Für die Kommunikation mit mehreren Layern sind in der  \ac{DCC} Architektur vier Komponenten definiert. Die Komponenten sind mit den \ac{DCC} Interfaces verbunden, die selber auf den Interfaces der Layer zugeordnet werden. Die Komponenten werden in den Layern erklärt, in denen sie liegen. 

Die DCCmgmt Komponente ist im Management Layer angeordnet. Sie übernimmt dort die Cross Funktionalität und steuert die anderen Komponenten. Dazu hat sie Unterkomponenten.

Eine Unterkomponente der DCCmgmt Komponente ist die DCC\_CROSS\_Facilities. 



\subsubsection{CI/ITS-S application mapping}
Das CI/ITS-S application mapping weist Anwendungen ein Kommunikationsinterface zu. Die Zuordnung geschieht durch einen Application Mapping Manager. Für eine  Zuordnung benötigt er verschiedene Angaben. Die wichtigsten Angaben sind welche Anforderungen die Application hat und welche Interfaces verfügbar sind, bzw. welche Eigenschaften und welchen Status sie haben. Zusätzlich muss bekannt sein, was die anwendbaren Regeln sind. 


\subsection{Security Layer \label{architektur_securityLayer}}
Der Security Layer ist ein Cross Layer der \ac{ITS} Architektur. Er sorgt für die Sicherheit im \ac{ITS} System. Im \autoref{architektur_dataSecurity} werden Eigenschaften der Datensicherheit und Mechanismen, die diese gewähren erläutert. Der Security Layer selber stellt diese  zur Verfügung. Damit die Data Security im gesamten System Anwendung finden kann hat er Interfaces in alle anderen Layer. 	

Der Standard \cite{en302665} beschreibt für den den Security Layer folgende Funktionalitäten:
\begin{itemize}
	\item Firewall und Angriff Management
	\item Authentifizierung, Autorisierung und Profilverwaltung
	\item Identitäts-, Krypto Schlüssel und Zertifikatsmanagement
	\item Eine allgemeine Sicherheit Informationsbasis (SIB)
	\item Hardware Security Module (HSM)
	\item Bereitstellung von Interfaces zu den anderen Layern, indem ihen Security Services angeboten werden
\end{itemize} 



\section{Data Security\label{architektur_dataSecurity}}
Der Begriff Security umfasst in \ac{ITS} verschiedene Bereiche der Sicherheit. Im \ac{ITS} Netz muss sichergestellt werden, dass die Nachrichten für die berechtigten Teilnehmer lesbar sind, für den unberechtigten Teilnehmer aber nicht. Dabei ist zu beachten, dass über \ac{ITS} sensible Daten verbreitet werden. Es muss neben dem Schutz gegen unbefugtes Lesen auch ein Mechanismus implementiert sein, der verhindert, dass falsche Informationen übertragen werden. Da die \ac{ITS} Architektur verschiedene Netzwerktypen \autoref{architektur_ueberblickNetzwerke} beinhaltet muss die Architektur auch ein Data Security Konzept beinhalten. Der Standard \cite{ts102940} spezifiziert dieses Konzept.  

\begin{figure}
	\includegraphics[width=0.95\textwidth]{content/images/02_architektur/securityLayer.pdf}
	\caption{Darstellung des Security Layers \cite{en302665}}
	\label{fig:architektur_securityLayer}
\end{figure}


Die \autoref{fig:architektur_securityLayer}  beschreibt die \ac{ITS-S} Architektur. Dabei wird besonders der Security Layer hervorgehoben und genauer beschrieben. In der Abbildung sind die Interfaces des Security Layers abgebildet. Es ist auch zu erkennen, dass der Security Layer in weitere vier Untermodule, bzw. Funktionalitäten aufgeteilt ist. Diese vier Untermodule sind, zumindest von dem \ac{ETSI} noch nicht abschließend beschrieben. So konnten beispielsweise zum Thema Firewall keine Dokumente gefunden werden. Die \autoref{fig:architektur_securityDienste} bietet einen weiteren Überblick über die verschiedenen Layer. Die dort beschriebenen Funktionalitäten sind die, die der Security Layer anbietet.  

\begin{figure}
	\includegraphics[width=0.99\textwidth]{content/images/02_architektur/securityDienste.pdf}
	\caption{Die ITS Layer mit Diensten, die Security nutzen \cite{ts102940}}
	\label{fig:architektur_securityDienste}
\end{figure}

\subsection{Trust and Privacy Management \label{architektur_TrustAndPrivacyManagement}}
Privacy beschreibt im Allgemeinen den Begriff Privatsphäre. Da \ac{ITS} auch zum Zwecke der Datenerhebung entwickelt wurde, mussten bereits am Anfang Konzepte entwickelt werden, die die eigentlich widersprüchlichen Forderungen von Datenerhebung und Datenschutz vereinbaren. Um zu verstehen, wie sie funktionieren muss der Begriff Privatsphäre unterteilt und genauer beschrieben werden. Die im Folgenden erläuterten Begriffe sind im Standard berücksichtigt. 

\subsubsection{Anonymity}
Der Begriff anonymity bedeutet übersetzt Anonymität und erklärt sich von alleine. Der Begriff bedeutet, dass jemandem anonymen keine Identität zugeordnet werden kann. Ein Bespiel hierfür ist eine völlig zufällige Nummer, die anstelle der Identität angegeben wird. Ihr kann keine Identität zugeordnet werden. Da einigen Services der \ac{ITS} Architektur eine Authentifizierung zu Grunde liegt, ist eine vollständige Anonymisierung nicht möglich. Aus diesem Grund gibt es die drei anderen Schlüsselwörter.

\subsubsection{Pseudonymity}
Pseudonymity bedeutet übersetzt Pseudonymisierung.  Pseudonymisierung wird oft mit Anonymisierung verwechselt, bedeutet aber etwas Anderes. Hinter etwas pseudonymisiertem steht eine Identität, die durch ein Pseudonym ersetzt wurde. Ein Beispiel hierfür ist die Matrikelnummer eines Studenten. Einem Angreifer, ohne die Möglichkeit, die Matrikelnummer dem Namen eines Studenten zuzuordnen, ist der Student gegenüber anonym. Besteht die Möglichkeit jedoch kann der Matrikelnummer sehr wohl eine Identität zugeordnet werden. Der Nachteil von Pseudonymen ist, dass sie wertlos sind, sobald ein Teil des Systems kompromittiert wurde. In \ac{ITS} wird die  Psyeudonymität erreicht, indem nur temporäre Kennungen von \ac{ITS-S} verwendet werden.


\subsubsection{Unlinkability}
Ein weiterer Nachteil von Pseudonymen ist, dass mit der erfassten Datenmenge die Chance steigt, von ihnen auf eine Identität zu schließen. Wird die Nutzung eines Pseudonyms verfolgt, können die Aktivitäten miteinander verlinkt werden. Dadurch kann aus dem Pseudonym eine neue Identität werden, von der Rückschlüsse auf die Identität hinter dem Pseudonym getroffen werden können. Es kann aber auch die Identität aus dem Verhalten ermittelt werden. Um bei dem Beispiel mit dem Studenten zu bleiben, kann über ihn, bzw. seine Matrikelnummer, ein Bewegungsprofil erstellt werden, das  einzigartig ist. Wird die Matrikelnummer für das Einschreiben in Kurse genutzt kann nach genug Einschreibungen das Studienfach und das Studiensemester ermittelt werden. Diese Sammlung nennt man Verkettung von Daten. Das Schlüsselwort unlinkability fordert, dass eine \ac{ITS-S} nicht verkettbar ist. In \ac{ITS} wird die Verkettbarkeit verhindert, indem die Verwendung von nicht, oder kaum, veränderter Informationen verhindert wird. Damit kann die Verbindung einer \ac{ITS-S} nicht anderen Verbindungen zugeordnet werden.

\subsubsection{Unobservability}
Unobservability bedeutet, dass der Nutzer eine Ressource nutzen kann, ohne dass andere Nutzer, oder Dritte, feststellen können, dass dieser Dienst genutzt wird.



Neben dem Datenschutz muss im \ac{ITS} System aber auch eine Zugangskontrolle realisiert werden. Die Zugangskontrolle teilt sich in zwei Bereiche auf: Die Berechtigung, das \ac{ITS} System als Ganzes zu nutzen und die Berechtigung, einzelne Services und Anwendungen zu nutzen. Die Prüfung der Identitäten wird über Zertifikate und Public-Key Verfahren realisiert. Zur Verteilung der Berechtigungen werden im Standard \cite{ts102731} definiert.









\subsection{Enrolement \label{architektur_enrolement}}
Die Funktion Enrolement authentisiert eine \ac{ITS-S} und ermöglicht ihr damit den Zugriff auf das  \ac{ITS} System.  Das bedeutet, dass die Identität der \ac{ITS-S} dem System gegenüber bestätigt wird. Dieser Zugriff ist noch keine Genehmigung, bestimmte Dienste nutzen zu können. Diese wird erst nach einer Autorisation, vgl. \autoref{architektur_authorization}, erteilt. Zuständig für die Regelung des Zugangs ins \ac{ITS} Netz ist eine Enrolment Authority.

Der Zugang wird durch Enrolement Authorities gewährt.

\subsubsection{Enrolment Authority}
Eine Enrolement Authority prüft die Identität einer \ac{ITS-S}. Wenn die Identität erfolgreich überprüft wurde  übermittelt die Enrolement Authority enrolement credentials an die \ac{ITS-S}. Enrolement credentials lassen sich mit \glqq Registrierungs Zeugnis\grqq~übersetzen. Sie können als digitale Zertifikate, die  mit kryptographischen Prüfsummen gesichert sind, ausgeteilt werden. Anhand dieser Zertifikate kann sich die \ac{ITS-S} als gültiger Teilnehmer des \ac{ITS} Systems ausweisen. So kann beispielsweise beim Versenden von \ac{CAM}, vgl. \autoref{sec:cam}, bewiesen werden, dass die sendende \ac{ITS-S} dazu berechtig ist und dass ihr vertraut werden kann. 

\subsubsection{Ausgetauschte Daten}
Um eine Registrierung zu ermöglichen muss die Enrolement Authority Daten von der \ac{ITS-S} erhalten. Hierbei sind nur die Daten aufgeführt, die relevant für das Verständnis des Systems sind. Steuerdaten, wie z.B. der Grund eines Fehlers, sind hier nicht aufgeführt. Der Austausch dieser Daten geschieht bereits verschlüsselt. 

\paragraph{Canonical Identity} Bei der Canonical Identity handelt es sich um eine weltweit eindeutige Kennung einer \ac{ITS-S}. Diese ist während der gesamten Produktlaufzeit gültig und kann zu einer Enrolement Authority gesendet werden, wenn die \ac{ITS-S} Enrolement Credentials anfordert. Die Canonical Identity soll als Oktett String dargestellt werden, also einer Sequenz von Bytes.

\paragraph{ITS-S Key}
Hierbei handelt es sich um einen Schlüssel. Der Key kann ein öffentlicher oder ein geheimer, bzw. symmetrischer, Schlüssel sein. 

\paragraph{List of enrolement credentials}
Diese Liste enthält die ausgestellten Enrolement Credentials. 

Es gibt verschiedene grundlegende Operationen auf die Enrolment Credentials. Diese stellen die Oberbegriffe für weitere Operationen da. 


 
\subsubsection{Obtain Enrolement Credentials \label{architektur_obtainEnrolementCredentials}}
Die Operation Obtain Enrolement Credentials bedeutet, dass eine \ac{ITS-S} die Credentials von einer Enrolement Authority erhält. 

Sie beginnt damit, dass die \ac{ITS-S} feststellt, dass sich für die \ac{ITS} Dienste anmelden muss. Aus diesem Bedürfnis wird ein Enrolement Request. Bei dem Enrolement Request handelt es sich um eine Anfrage an eine Enrolement Authority. Diese fragt bei einer Authenticate Authority an, ob die \ac{ITS-S} für die \ac{ITS} Dienste berechtigt ist und stellt der \ac{ITS-S} im positiven Fall einen Satz Enrolement Credentials aus und übermittelt diese an die \ac{ITS-S} .  

\begin{figure}[h]
	\includegraphics[width=0.75\textwidth]{content/images/02_architektur/enrolementFunktionaleKomponenten.pdf}
	\caption{Das funktionale Modell der Enrolement Credentials \cite{ts102731}}
	\label{fig:architektur_enrolementFunktionalModel}
\end{figure}

Die \autoref{fig:architektur_enrolementFunktionalModel} beschreibt den Prozess Obtain Enrolement Credentials mit einem erfolgreichen Ausgang. Sie steht stellvertretend für die anderen Prozesse, deren Abläufe sich kaum ändern. In der Abbildung sind funktionale Elemente definiert. Sie werden folgendermaßen beschrieben:
\paragraph{Invoke Enrolment} Dieses funktionale Element entdeckt das Bedürfnis einer \ac{ITS-S}, dass sie sich an der \ac{ITS} Infrastruktur anmelden muss und initiiert die Anmeldungsprozedur.

\paragraph{Enrolment Request} Dieses funktionale Element liefert die Anmeldungsanfrage für einer \ac{ITS-S} an die \ac{ITS} Infrastruktur. Anschließend empfängt es die entsprechende Antwort und speichert die geheimen Kommunikationsparameter in der \ac{ITS-S}.


\paragraph{Enrole Station} Dieses funktionale Element empfängt eine Anmeldungsanfrage einer \ac{ITS-S}. Nach dieser Anfrage beginnt es, den Benutzer zu authentisieren. Wenn diese Authentisierung erfolgreich war, sendet es eine positive Antwort mit den Parametern, die für eine laufende sichere Kommunikation nötig sind, an die \ac{ITS-S}. Die Enrolement Station entspricht der Enrolement Authority. In diesem Fall ist sie als funktionales Element beschrieben. Das bedeutet, dass es in diesem Modell nur dieses eine Element gibt. 

\paragraph{Authenticate Station} Dieses funktionale Element validiert die Identitäts\-informationen, die von dem funktionalen Element Enrol Station empfangen werden. Hierbei handelt es sich um die Authentication Authority. Sie ist aber als funktionales Element beschrieben.

\paragraph{Process Authentication} Dieses funktionale Element enthält die Antwort auf die Abfrage des funktionalen Elements  Authentication Authority .
 
 
\subsubsection{Update Enrolment Credentials}
Die Enrolement Credentials können aktualisiert werden. Dies geschieht auf die Anfrage einer \ac{ITS-S}. Initiiert wird die Aktualisierung, wenn die \ac{ITS-S}  einen Bedarf feststellt. Der weitere Prozess entspricht dem Anfordern von Enrolement Credentials, vgl. \autoref{architektur_obtainEnrolementCredentials}.

\subsubsection{Remove Enrolment Credentials}
Der \ac{ITS-S} Service kann einer \ac{ITS-S} die Enrolement Credentials wieder entziehen, bzw. diese Enrolement Credentials für ungültig erklären. Dadurch werden alle Informationen, die die \ac{ITS-S} verteilt, auch ungültig. Damit dieses Verfahren effektiv ist, müssen die Enrolement Authority und die Authorization Authority miteinander kommunizieren. Dadurch wird sichergestellt, dass korrekte Autorisierung Status Updates verteilt werden. 

Der Prozess beginnt, wenn die Enrolement Authority feststellt, dass das die Liste mit den Enrolement Credentials aktualisiert werden muss. Dabei wird die Enrolement Credentials aus dem gesamten \ac{ITS} Netzwerk entfernt. Das Entfernen beginnt bei der  Enrolement Authority, sie entfernt die Credentials aus ihrer Liste. Anschließend macht sie bei den    Authorization Authority bekannt, dass die entsprechenden Credentials ungültig sind und broadcastet an alle Teilnehmer im \ac{ITS-S} Netz, dass die Credentials ungültig sind.


\subsection{Authorization \label{architektur_authorization}}

Autorisieren bedeutet, dass Rechte eingeräumt werden. Im Sinne von \ac{ITS-S} bedeutet es, dass eine \ac{ITS-S} durch die Autorisierung bestimmte Services nutzen darf. Die Voraussetzung für eine Autorisierung ist eine Authentisierung, vgl. \autoref{architektur_enrolement}. 

Die Autorisierungen werden von den Authorization Authority verwaltet.
\subsubsection{Authorization Authority}
Authorization Authorities verwalten die Berechtigungen von \ac{ITS-S}. Sie  sollen hierarchisch aufgebaut und für verschiedene Kontexte verfügbar sein. Ein Kontext grenzt sich durch die  Applications, die er autorisiert, eine Zeitdauer, eine geographische Region oder andere  kodierbare Gründe ab. Die Aufteilung der Applications kann sehr granular sein, es können beispielsweise für die einzelnen Einträge von \ac{CAM} (\autoref{sec:cam}) unterschiedliche Authorization Authority zuständig sein. Der hierarchische Aufbau bedeutet, dass es verschiedene Level von Authorization Authorities geben kann. Higher Level Authorization Authorities sollen lower-level Authorization Authorities autorisieren, diese autorisieren dann die \ac{ITS-S}. 

\subsubsection{Obtain Authorization Tickets \label{architektur_obtain-authorisation-tickets}}
Obtain Authorization Tickets beschreibt das Empfangen von Authorisation Tickets. Dieser Prozess wird von den \ac{ITS-S} ausgelöst. Dazu fragt die \ac{ITS-S} einer Authorization Authoritiy wegen einem Authorisation Ticket an. Diese Anfrage beinhaltet bereits die Enrolement Credentials der \ac{ITS-S}. Diese überprüft anhand der Enrolement Credentials die Identität der \ac{ITS-S} und beantwortet die Anfrage. Die Antwort kann entweder  Authorization Tickets enthalten, oder den Grund, warum die \ac{ITS-S} keine bekommt.


\subsubsection{Update Authorization Tickets}
Autorisation Tickets haben nur eine begrenzte Lebensdauer. Diese Lebensdauer ist von der Application abhängig. Aus diesem Grund müssen sie regelmäßig erneuert werden. Wenn der Bedarf besteht, die Authorization Tickets zu aktualisieren, fordert die \ac{ITS-S} von einer Authorisation Authory eine Aktualisierung an. Die benötigten Daten entsprechen dem Anfordern von Tickets, vgl. \autoref{architektur_obtain-authorisation-tickets}.

\subsubsection{Publish Authorization Status}
Der Authorisation Status wird versendet, wenn das Versenden explizit von einer \ac{ITS-S} oder einer Authorisation Authory angefordert wird. Der Grund hierfür kann sein, dass einer \ac{ITS-S} Berechtigungen entzogen werden sollen. Dabei können einzelne Berechtigungen oder alle temporär oder dauerhaft entzogen werden.

\subsubsection{Update Local Authorization Status Repository}
Fall eine \ac{ITS-S} keinen Zugang zu der Infrastruktur hat, nutzt sie einen lokalen Autorisierungsspeicher um die Athorisierungsinformationen von anderen \ac{ITS-S} zu überprüfen. Sobald die \ac{ITS-S} wieder Zugang zur \ac{ITS-S} Infrastruktur hat, wird der Speicher aktualisiert.  

\subsection{Interfaces}
Der Security Layer besitzt Interfaces in alle anderen Layer. Diese Interfaces stellen die Kommunikationsschnittstellen dar. Die genaue Spezifizierung dieser Interfaces findet zum aktuellen Zeitpunkt noch statt. Die Standards, die die  Interfaces genauer beschreiben sind noch im  \glqq Draft\grqq , bzw. \glqq early Draft\grqq , Zustand und damit nicht verfügbar. 

\subsection{Angebotene Services}
 \begin{longtable}{| p{0.24\linewidth} | p{0.24\linewidth} | p{0.24\linewidth} |p{0.24\linewidth}|}
 \hline
 \textbf{Maßnahme} & & \textbf{Security Services} &  \\
 \cline{2-4}
 &\textbf{ First Level} & \textbf{Lower Level} & \textbf{Data Accessed}\\
 \hline
Include pseudonym in all V2V messages & Pseudonym Validation & & \\
 \hline
 Require an ITS-S to be authorized by an ITS authority before its messages are accepted by the ITS system & Obtain Enrolment Credentials & & Security Parameters (Authentication Keys)\\
 \hline
 Limit message traffic to V2I/I2V where possible & Obtain Enrolment Credentials & & Security Parameters (Authentication Keys) \\ 
 \cline{3-4}
 & & Authorization & Policy Database, Security Parameters (Authorization Ticket) \\
 \cline{3-4}
& & Establish Security Association & Security Parameters (Pseudonym, Encryption Keys) \\
 \cline{2-4}
 & Send Secured Message & Encrypt Outgoing Message & Security Parameters (Pseudonym, Encryption Key) \\
  \cline{3-4}
 &  & Authenticate Outgoing Message & Security Parameters (Pseudonym, Authentication Key) \\
 \cline{2-4}
 & Receive Secured Message & Decrypt Incoming Message & Security Parameters (Encryption Key) \\
  \cline{3-4}
 & & Validate Authentication on Incoming Message & Security Parameters (Pseudonym, Authentication Key) \\
 \cline{2-4}
 & Update Security Association & Remove Security Association & Security Parameters (Pseudonym, Encryption Key) \\
 \cline{3-4}
 & & Establish Security Association & Security Parameters (Pseudonym, Encryption Key) \\
 \cline{2-4}
 & Remove Enrolment Credentials & Authorization & Policy Database, Security Parameters (Authorization Ticket) \\
 \cline{3-4}
 & & Remove Security Association & Security Parameters (Pseudonym, Encryption Key) \\
 \hline
 Implement plausibility validation on incoming information & Validate Data Plausibility & Validate Dynamic Parameters & LDM \\
 \cline{3-4}
 & & Validate Timestamp & \\
 \cline{3-4}
 & & Validate Sequence Number &  \\
 \hline
 Include a non cryptographic checksum of the message in each message sent  & Insert Check Value & Calculate Check Value & \\
 \cline{2-4}
 & Validate Check Value & Calculate Check Value &  \\
 \hline
 Use broadcast time (Universal Coordinated Time - UTC - or GPS) to timestamp all messages & & Timestamp Message & \\
 \cline{3-4}
 & & Validate Timestamp & \\
 \hline
 Include a sequence number in each new message & & Insert Sequence Number & \\
  \cline{3-4}
 & & Validate Sequence Number & \\
 \hline
 Include an authoritative identity in each message and authenticate it & Validate pseudonym & & Security Parameters (Authentication Keys) \\
 \hline
Encrypt the transmission of personal and private data &  Send Encrypted Data & Encrypt Outgoing Message & Security Parameters (Encryption Keys) \\
\cline{2-4}
& Process Received Encrypted Data & Decrypt Incoming Message & Security Parameters (Encryption Keys) \\
\hline
Add an audit log to ITS stations to store the type and content of each message sent to and from an ITS-S &  Update Audit Log & Record Incoming ITS Messages& Audit Logs\\ \cline{3-4}
& & Record Outgoing ITS Messages & Audit Logs \\
\hline
Digitally sign each message using a Kerberos/PKI-like token & Sign Outgoing Message & Generate Signature & Security Parameters (Certificate, Keys) \\ 
\cline{3-4}
& &  Authorization & Policy database, Security Parameters (Authorization Ticket) \\
\cline{2-4}
& Verify Incoming Signed Message & Verify Signature & Security Parameters (Certificate, Keys) \\
\cline{3-4}
& & Authorization  & Policy database, Security Parameters (Certificate Status Information) \\
\hline
Use a pseudonym that cannot be linked to the true identity of either the user or the user's vehicle & Obtain Enrolment Credentials & Identification (authoritative identity provider) & Security Parameters (Pseudonym, Encryption Key) \\
\cline{2-4}
& remove Enrolment Credentials & Identification (authoritative identity provider) & Security Parameters (Pseudonym, Encryption Key)\\
\hline
Allow remote activation and deactivation of ITS-S & ITS-S Remote Management Report Misbehaving ITS-S & Authorization & Policy Database \\
\cline{3-4}
 & & Deactivate ITS Transmission & Security Parameters (Authorization Ticket) \\
 \cline{3-4}
& &  Activate ITS Transmission & Security Parameters (Authorization Ticket) \\
\cline{3-4}
& & Report Misbehaviour & Security Parameters (Authorization Ticket) \\
\hline 
\caption{Tabelle mit den Sicherheitsmaßnahmen \cite{ts102731}}
\label{tab:architektur_tabelleSicherheitsmassnahmen}
 
 \end{longtable}
\autoref{tab:architektur_tabelleSicherheitsmassnahmen} stammt aus dem Standard \cite{ts102731}. Dort werden die Maßnahmen zusammengefasst, die benötigt werden um eine Sicherheit zu erreichen. \glqq First Level\grqq~sind die Security Services, die direkt von den Anwendungen oder anderen Komponenten aufgerufen werden.  \glqq Lower Level\grqq~Services sind die, die von anderen Security Services aufgerufen werden.


\subsection{ITS Security Serivces}
Bei den in \ac{ITS} versendeten Nachrichten müssen aus Sicht der Sicherheit drei verschiedene Typen betrachtet werden. 

Der erste Typ ist die \glqq Individual public message\grqq~Dieser Typ wird per Broadcast an andere \ac{ITS-S} versendet. Da es sich um eine Broadcast Nachricht handelt, ist eine Verschlüsselung nicht nötig. Um zu verhindern, dass mit dieser Nachricht Fehlinformationen übertragen werden muss sie lediglich signiert werden. Der Standard \cite{ts102731} nennt für diesen Typ von Nachricht die Schlagwörter: authorization, authentication und integrity. Damit wird ausgedrückt, dass die sendende \ac{ITS-S} im System legitimiert sein muss, die Nachricht wird als Schutz gegen Verfälschungen und fehlerhafte Nachrichten von Angreifern aber signiert.

Der zweite Typ ist die \glqq    Individual private message\grqq~. Sie wird an eine bestimmte \ac{ITS-S} verschickt. Da hier kein Broadcast vorliegt, macht es auch Sinn diese Nachricht zu verschlüsseln. Der Standard nennt zu diesem Nachrichtentyp die Schlüsselwörter require authorization, authentication, integrity, privacy und confidentiality. Diese Schlüsselwörter erweitern die Schlüsselwörter der \glqq Individual public message\grqq~um den Faktor, dass der Inhalt von Dritten nicht mitgelesen darf. Aus diesem Grund findet zusätzlich eine Verschlüsselung der Nachricht statt.

Als drittes sind die \glqq Security Associations\grqq zu nennen. Sie werden zwischen zwei oder mehreren \ac{ITS-S} aufgebaut und beinhalten einen Satz von Krypto Algorithmen, Schlüsseln und andere private und öffentliche Parameter. Sie werden genutzt um eine Ende zu Ende Verbindung aufzubauen.  Der Standard nennt die Schlüsselwörter confidentiality, authentication und integrity.  Diese entsprechen prinzipiell den Schlüsselwörtern der \glqq    Individual private message\grqq. Diese Verbindungen können auf andere autorisierte Teilnehmer dupliziert werden. Jeder Teilnehmer kann mehrere sichere Verbindungen haben. Wenn de sichere Verbindung aufgebaut wird, sollen die Zertifikate und die Kryptowerkzeuge die mit den sicheren Verbindungen verknüpft sind genutzt werden. Die sichere Verbindung soll regelmäßig neu verhandelt werden. 

\begin{figure}[h]
	\includegraphics[width=0.95\textwidth]{content/images/02_architektur/enrolementScenario-ts102_940.pdf}
	\caption{Ein beispielhaftes Zusammenspiel der Security Komponenten \cite{ts102940}}
	\label{fig:architektur_enrolementBeispielSzenario}
\end{figure}

