%=========================================
% 	   Einleitung     		 =
%=========================================

\chapter{Einleitung}

\section{Latex installieren und einrichten}
\subsection{Unter Windows}

Als Latexdistribution unter Windows steht \href{http://www.miktex.org/}{\textit{MikTex}} zu Verf�gung, die als freie Software im Internet erh�ltlich ist. 
\textit{MikTex} unterst�tzt Windows XP, Vista und Windows 7. Neben \textit{MikTex} wird noch ein PostScript-Interpreter ben�tigt, 
z.B. GhostScript, zu finden auf \href{http://www.chip.de}{Chip.de}.

\textit{Wichtig:} Bei \textit{MikTex} unbedingt Vollinstallation ausw�hlen, sonst sind eventuell ben�tigte Packages nicht vorhanden.

\subsection{Unter Linux}

Unter Linux existiert die Latexdistribution \textit{texlive}, die als aktuelle Version aus den Paketquellen geladen werden kann (unter Ubuntu mit 
\lstinline{apt-get install texlive-full}). Auch hier ist ganz wichtig, die volle Distribution zu laden, damit alle Packages zur Verf�gung stehen.

\section{Entwicklungsumgebungen}

Hat man die passende Distribution installiert, bieten sich vielerlei M�glichkeiten an ein Latex-Projekt anzugehen oder einzelne Dokumente zu editieren. Unter
Windows k�nnten dies folgende sein:

\begin{description}
 \item [TexnicCenter] Umfangreiche Entwicklungsumgebung mit Projektorganisation und Autovervollst�ndigung
 \item [TexLipse] Eclipse-Plugin, das alle Vorteile der Eclipseumgebung mit Latex verbindet
 \item [Texmaker] Einfacher Latexeditor mit Pdf-Direktvorschau
\end{description}


Unter Linux stehen bereit:

\begin{description}
 \item [Gummi] Ebenfalls einfacher Latexeditor mit Direktvorschau
 \item [TexLipse] Auch f�r Linux erh�ltlich
 \item [Kile] Umfangreiche Entwicklungsumgebung, �hnlich wie TexnicCenter
\end{description}

Nach der Installation muss die Entwicklungsumgebung eingerichtet werden; dazu finden sich viele Anleitungen im Internet, die genau erkl�ren, welche Distribution
auf welche Weise eingerichtet wird. Insbesondere sollte der PDF-Viewer festgelegt werden, damit bei Gummi und Texmaker die Direktvorschau funktioniert. Manchmal kommt es vor, dass die Ausgabe 
nach dem Kompilieren Umlaute und Sonderzeichen nicht richtig darstellt. Unter Linux h�ngt dies mit den unterschiedlichen Zeichens�tzen zusammen, die unterst�tzt
werden. Um diese Vorlage zu verwenden ist es notwendig den verwendeten Zeichensatz des Editors bzw. der Entwicklungsumgebung auf den in diesem Dokument
verwendeten Zeichensatz - \textit{ISO-8859-9} - umzustellen.

\section{Werkzeuge}

\begin{description}
 \item [\href{http://jabref.sourceforge.net/}{JabRef}] Ein Literaturverwaltungsprogramm, welches das \textit{BibTeX}-Format einsetzt
 und mithilfe einer graphischen Oberfl�che das Anlegen von Literaturverzeichnissen vereinfacht.
 
\end{description}

\section{Einrichtung und Konfiguration der Vorlage}

\subsection{Struktur der Vorlage}
%Fehlt noch; erst, wenn Struktur endg�ltig feststeht.

\subsection{Die Config-Datei}

\begin{itemize}
 \item Datei/Struktur erkl�ren
 \item Config-Datei
\end{itemize}


